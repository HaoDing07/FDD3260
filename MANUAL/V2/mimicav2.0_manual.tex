%%%%%%%%%%%%%%%%%%%%%%%%%%%%%%%%%%%%%%%%%%%%%%%%%%
% ENTETE                      %%%%%%%%%%%%%%%%%%%%
%%%%%%%%%%%%%%%%%%%%%%%%%%%%%%%%%%%%%%%%%%%%%%%%%%

\documentclass[11.5pt,A4,french]{article}
\usepackage{graphics}
\usepackage{hyperref}
\usepackage{epsf}
\usepackage{amsmath}
\usepackage{epsfig}
\usepackage{amssymb}
\usepackage{epic}
\usepackage{array}
\pagenumbering{arabic}
\usepackage{aeguill}
\usepackage{geometry}
\usepackage{fancyhdr}
\usepackage{lscape}
%%%%%%%%%%%%%%%%%%%%%%%%%%%%%%%%%%%%%%%%%%%%%%%%%%%%%%%%%%%%%%%%%%%%%%%%%%%%%%
%\pagenumbering{arabic}
%\pagestyle{fancy}
%\fancyhf{}
\lhead{}
\rhead{}
\geometry{hmargin=1.8cm,vmargin=1.7cm}
\renewcommand{\baselinestretch}{1.03}
%\fancypagestyle{plain}{
%\fancyhead{}
%\renewcommand{\headrulewidth}{0pt}}
%%%%%%%%%%%%%%%%%%%%%%%%%%%%%%%%%%%%%%%%%%%%%%%%%%%%%%%%%%%%%%%%%%%%%%%%%%%%%%

\begin{document}

\begin{titlepage}

\voffset=10 cm

\begin{center}
{\bf\huge{MIMICA V2.0, User guide}}\\
\vspace{3 cm}
\large{Julien SAVRE}\\
\large{31-01-2013}
\end{center}

\end{titlepage}

\tableofcontents
\newpage

\section{Coding standards}

The MIMICA model has been designed as an extension of the CRM-MIT model developed by Chien Wang and Julius Chang \cite{WC1993}. As such, it
still contains many of the features originally made available in the CRM-MIT and most of the variable names were retained. Not considering the additions
regarding the numerics and physics implemented, the present LES code differs mostly from its predecessor in the coding standards adopted, the overall structure
of the code being kept as close as possible to the original CRM. 

Note that even though most of the model's core routines have already been modified to fully comply with the following standards, some of the optional modules
(aerosol and chemistry) as well as certain routines generally used as black boxes have just been partially rewritten so far. As a 
result, a certain inhomogeneity in the style exists between the main routines used for physical and numerical calculations and several peripheric packages and
modules.

The standards adopted try to optimize as much as possible the clarity of the code by making use of modern fortran 90/95 commands, and by adding commented
lines where this seems necessary. This is sometimes done to the detriment of code performances and overall optimization, but we believe that this drawback
is largely compensated by the simplicity of comprehension and easiness of coding made available to the user. The most commonly used routines of the code have 
been rewritten in REAL fortran 90 which enables a more condensed writting (vector operations). 

The basic rules used in the new fortran 90 routines are described below:
\begin{itemize}
\item each routine or package is introduced by a header containing a short description of the routine and the history of all the major changes made since
the first version;
\item all the variables used in the calculations (winds, hydrometeors, turbulent parameters, thermodynamics....) must be invoked through the use of the
corresponding modules (see section \ref{structures}), followed by {\it implicit none};
\item local variables are defined in the local routine only; variables called in several calculations or written in output files must be defined in one of the
data structure/type ({\it shared\_data} for constants, {\it shared\_xxxxx} for 3D fields); 
\item the use of {\it \#include} statements in the declaration section must be avoided as much as possible;
\item {\it if}, {\it do}, etc... statements are idented: 2 void spaces are left before each new nested block;
\item if a code line is cut and appears on several lines, 4 void spaces are added before the following lines;
\item no tabulation, use the space bar as much as possible;
\item add commented lines introduced in the first column by the character "!" (for example to introduce new sections or delimit independent operations);
the comments must be short but explicit enough to describe the purpose of the following lines of code;
\item the original models must be left untouched: if new methods are implemented, create an optional environment variable in {\it start} and {\it wheader.csh};
\end{itemize}

MIMICA is managed through the subversion system SVN. This software provides a very simple way to share our developments and get the latest updates of the code. 
This makes the developer's life much easier and allows everyone to follow real-time all the current developments and latest modifications, with an improved
possibility to track bugs or upgrade its personal version. Basically, using SVN, a common main repository for the code is stored somewhere which includes in a
condensed format all the successive versions of all the files. This repository is managed by an administrator fixing the access rules and its initial content.
All the registered users having downloaded the code through SVN can either download the current version stored in the remote repository or, on the
opposite, put their own local modifications into the main repository. SVN will automatically merge every modifications. Therefore, there always exist one
local version of the code owned by each user that can be managed and customized at will, and one remote version including all the latest commited modifications
(plus we could add all the previous commited versions stored into the remote directory). It is strongly recommended that users make sure the remote
repository always contains a clean and running version of the code (that means the code must compile, run properly -no segmentation fault for example-, that there
are no "print*" for debugging, no case specific branch or bypass, etc.).

As a brief introduction to SVN, here are the most basic commands to know:
\begin{itemize}
\item the code can be downloaded through: {\it svn checkout ssh://x\_julsa\@vagn.nsc.liu.se/svn\_MIMICA .}
\item when a working modification has been done, the development can be shared through: {\it svn commit file1.f90 file2.f90 ...}; an editor window will open
automatically in which a short description of the commit can be written
\item in order to update its own version and merge it with the main repository to account for the latest modifications, simply type: {\it svn update}
\item differences between its own version and the main repository can be displayed via: {\it svn diff}.
\end{itemize}
Of course, more can be done thanks to SVN. As all the successive versions of each routine is stored, it is possible to apply these commands to one
particular release of the code by adding the option {\it -rxxx}, where xxx is the release number. To perform the previous operations, a personal password will be 
required. In order for every user to follow the evolution of the code, an automatic mail is sent to the subscribers when a commit is done: the mail contains
some important information such as the date, the author's name, the modified files and the short description written by the author.

\newpage
\section{Physical model description}

In the following section, the description focuses on the main core of the model which is necessarily used in each application. The aerosol and chemistry modules are
therefore excluded (temporarily). 

\subsection{The LES approach}

In LES, the governing equations are filtered in the physical space. Depending on the kind of spatial filter used, part of the turbulent spectrum, associated with the 
large turbulent eddies (small wave numbers), can be directly and accurately resolved. On the other hand of the turbulent spectrum, the small dissipative turbulent 
structures (large wave numbers), typically smaller than the cut-off scale of the filter, cannot be resolved. Subgrid scale (SGS) models thus need to be included in
the equations in order to account for the influence of the unresolved part of the spectrum on the resolved turbulent structures. 

On a mathematical point of view, the filter kernel is commonly denoted by $G_{\Delta}$, and the filtering operation can then be written as:
\begin{equation}
\overline{\Phi} = \int_{-\infty}^{+\infty} \Phi\left({\mathbf x}\right) G_{\Delta}\left({\mathbf x} - {\mathbf x^{\prime}}\right) d{\mathbf x^{\prime}},
\label{filter}
\end{equation}
where $\Phi$ is a dummy variable, ${\mathbf x}$ the position in the multidimensional space and $\overline{.}$ represents a filtered quantity. The kernel filter can
be given any prescribed shape in the physical or spectral space (top-hat, gaussian...). In practical LES codes, it is always more convenient to define implicitly the
filter which avoids the necessity to calculate integral \ref{filter}. The cut-off scale in the physical space is then defined both by the grid and by the numerical 
scheme (the dissipative role played by the numerical discretization is often omitted).

Based on the filtering operation \ref{filter}, any local, instantaneous quantity $\Phi$ can be decomposed in a resolved filtered part and an unresolver SGS part:
\begin{equation}
\Phi = \overline{\Phi} + \Phi^{\prime}.
\end{equation}
Note that two important properties that are true for RANS methods, are no longer valid in LES:
\begin{equation}
\overline{\overline{\Phi}} \neq \overline{\Phi} \mbox{\hspace{2cm} and \hspace{2cm}} \overline{\Phi^{\prime}} \neq 0.
\end{equation}
These two inequalities complexify greatly the filtered equations.

If we now consider that the evolution of $\Phi$ is governed by an advection/local production equation,
\begin{equation}
\frac{\partial \rho \Phi}{\partial t} + \frac{\partial \rho u_{i}�\Phi}{\partial x_{i}} = \dot{s},
\end{equation}
where $\dot{s}$ represents a local source or sink, and if we apply a spatial filter to this equation, we obtain:
\begin{equation}
\frac{\partial \overline{\rho \Phi}}{\partial t} +�\frac{\partial \overline{u_{i}} \overline{\rho\Phi}}{\partial x_{i}} = \overline{\dot{S}} + \mathcal{R}_{\Phi}.
\label{filtered}
\end{equation}
To simplify further this equation, Favre filtering can be invoked, where $\overline{\rho \Phi} = \overline{\rho }�\widetilde{\Phi}$. In equation \ref{filtered},
$\mathcal{R}_{\Phi}$ contains all the various cross correlations between $\Phi$ and $u$ appearing when the filter is applied. It reads:
\begin{eqnarray}
\mathcal{R}_{\Phi} & = & \frac{\partial}{\partial x_{i}}\left(\overline{\rho u_{i} \Phi} - \overline{u_{i}}\overline{\rho\Phi}\right) = \frac{\partial
\tau^{\Phi}_{i}}{\partial x_{i}}.
\end{eqnarray}
The exact form of $\mathcal{R}_{\Phi}$ is made complicated by the two properties introduced previously and won't be detailed here. It contains three main terms:
one represents the interactions between the large scales, one represents the interactions between small and large scales, and the last one contains only informations
on the subgrid scales. The turbulence closures discussed in section \ref{SGS} model the SGS tensor ${\mathbf \tau^{\Phi}}$.

\subsection{Governing equations}

The model solves for a non-hydrostatic incompressible system, in two or three dimensions. Compared to the original CRM model, the anelastic assumption constitutes one
of the major differences which is reflected by a heavily modified solver algorithm (see section \ref{numerics}). The equations are spatially filtered as described in
the previous section but the filter notation $\overline{.}$ will be omitted for clarity, as this is commonly the case in the litterature. 

\subsubsection{Pressure decomposition}
Using the anelastic, non-hydrostatic framework, the Exner pressure (used instead of the regular pressure) is decomposed into three parts (see for instance
Clarks \cite{Cla1979}): an isentropic reference state varying with altitude only, an horizontal deviation from the reference state assumed to be in hydrostatic 
balance with its environment, and a 3D dynamic perturbation which needs to be updated after each time step. This yields:
\begin{equation}
\Pi\left(x,y,z\right) = \pi\left(t;x,y,z\right) + \pi_{0}\left(z\right) + \pi_{1}\left(t;z\right),
\end{equation}
where $1$ denotes the hydrostatic part and $0$ the reference isentropic part. In the above relation, the adimensional Exner pressure was used instead of the 
thermodynamic pressure $P$ as commonly done in atmospheric models. Let's recall the relation between the two pressures:
\begin{equation}
\Pi = \left(\frac{P}{P_{ref}}\right)^{R/c_{p}},
\end{equation}
where $P$ is the total thermodynamic pressure; $P_{ref} = 100000$ $Pa$, $R$ is the ideal gas specific constant for dry air ($287$ $J/K/kg$) and $c_{p}$ is the specific
heat capacity at constant pressure for dry air ($= 1004$ $J/kg$). The same decomposition as for $\Pi$ can be applied to $P$.

The reference pressure $\pi_{0}$ is held constant throughout the simulation and does not require any update. It satisfies the following relation:
\begin{equation}
\frac{\partial \pi_{0}}{\partial z} = - \frac{g}{c_{p} \theta_{00}},
\label{p0}
\end{equation}
where $\theta_{00}$ is a reference potential temperature, usually chosen as being the surface value of $\theta$. 

The hydrostatic pressure deviation $\pi_{1}$ won't be evaluated in the same way if a large scale horizontal divergence is enforced ("traditional LES"), or if we consider a
free simulation. In the first case, $\pi_{1}$ is used to force the horizontal large scale divergence: $D_{LS} = -\frac{\partial w}{\partial z}$. If we decompose the vertical 
velocity as $w = w_{0} + w^{\prime}$, with $w_{0}$ the horizontal mean of $w$ and $w^{\prime}$ the deviation of $w$ around $w_{0}$, then the vertical wind tendency, including 
advection, SGS mixing and buoyancy, may result in a vertical acceleration of the horizontal mean. In order to keep this average constant and equal to the large scale forcing
$w_{LS}$ (determined by $D_{LS}$), we must impose an additional constraint:
\begin{equation}
\left<\frac{\partial w}{\partial t}\right> = \left<w_{s}\right> = 0.
\end{equation}
In the above, $w_{s}$ is the total vertical wind tendency and $\left<.\right>$ represents the horizontal average. In order to force the previous constraint,
$\pi_{1}$, appearing in $w_{s}$ as a vertical pressure gradient, is calculated to balance the vertical acceleration resulting from all the other physical phenomena. Noting 
$w_{s}^{*}$ the tendency without the $\pi_{1}$ gradient, we may write:
\begin{equation}
\left<w_{s}^{*}\right> - c_{p}\theta_{00}\frac{\partial \pi_{1}}{\partial z} = 0,
\end{equation}
or:
\begin{equation}
\frac{\partial \pi_{1}}{\partial z} = \frac{1}{c_{p}\theta_{00}} \left<w_{s}^{*}\right>,
\end{equation}
By integrating this last relation over the vertical dimension, $\pi_{1}$ can be updated after each time step so that the horizontal average of the vertical wind always 
satisfies the fixed horizontal flow divergence constraint. 

In the case of a free simulation, without any forcing, $\pi_{1}$ remains constant throughout the simulation and expresses simply the deviation of $\pi_{0}$ from the local
hydrostatic balance, so that we can write:
\begin{equation}
\frac{\partial \left(\pi_{0} + \pi_{1}\right)}{\partial z} = - \frac{g}{c_{p} \theta_{v0}}.
\end{equation}
Here, $\theta_{v0}$ is the initial virtual temperature sounding, varying with altitude. $\pi_{1}$ can the be calculated by substracting the previous relation with equation
\ref{p0}. The virtual potential temperature is defined by $\theta_{v} = \theta\left(1 + 0.61 q_{v}\right)$ where $q_{v}$ is the water vapor mixing ratio. The potential
temperature appearing in the previous relation is defined by $\theta = T\left(p/p_{ref}\right)^{R/c_{p}}$.

Finally, under the anelastic assumption, the dynamical pressure is updated after each time step to force the flow divergence to remain equal to 0. As explained in the next 
section, this requires the inversion of a Poisson equation derived from the assumption that $div\left(\mathbf{u^{n+1}}\right) = 0$ and for which $\pi$ is solution. 

\subsubsection{Balance equations}
The system includes the momentum equations, an energy equation (potential temperature equation), an equation for the total water mixing ratio and equations
for two hydrometeor moments: the number concentration and mass mixing ratio (see section \ref{micro}). The momentum equations in non conservative form read:
\begin{equation}
\frac{\partial {\mathbf u}}{\partial t} + u_{i}\frac{\partial {\mathbf u}}{\partial x_{i}} = - c_{p} \theta_{00}\frac{\partial \pi}{\partial x_{i}} + b\delta_{i,3} 
+ f_{k} \left({\mathbf u}-{\mathbf u_{g}}\right)\epsilon_{i,j,k} + \frac{1}{\rho_{0}}\frac{\partial {\mathbf \tau^{u}}}{\partial x_{i}}.
\label{momentum}
\end{equation}
In the above, ${\mathbf u}$ is the velocity vector $\left(u,v,w\right)$, $Cp$ is the atmosphere's heat capacity at constant pressure ($\approx 1005$), 
$\theta_{00}$ is a reference potential temperature (initial surface value), $b$ is the vertical buoyancy flux, ${\mathbf \tau^{u}}$ is the subgrid scale viscous tensor, 
and $f_{k}$ is the Coriolis parameter with ${\mathbf u_{g}}$ being the geostrophic wind vector. $\delta_{i,3}$ and $\epsilon_{i,j,k}$ denote the Kronecker and 
Levi-Civita symboles respectively whereas $\rho_{0}$ is the reference density of the isentropic state. The buoyancy flux, appearing only in the vertical wind equation, 
is expressed under the anelastic assumption and following Clarks as the deviation of the virtual potential temperature from its horizontal average:
\begin{equation}
b = g\left(\frac{\theta - \left<\theta\right>}{\theta_{00}}+ 0.61\left(q_{v} - \left<q_{v}\right>\right) - q_{l} - q_{i}\right),
\end{equation}
with $q_{l}$ and $q_{i}$ the total liquid and total ice mass mixing ratios. 

In the MIMICA model, a governing equation for $\theta_{il}$, the ice/liquid potential temperature, is solved as a proxy for the energy balance equation. 
The choice to use the ice/liquid potential temperature instead of any other temperature was made because $\theta_{il}$ is conserved during all the
microphysical processes. Its balance equation reads:
\begin{equation}
\frac{\partial \theta_{il}}{\partial t} + u_{i}\frac{\partial \theta_{il}}{\partial u_{i}}
= \frac{1}{\rho}\frac{\partial {\mathbf \tau^{\theta}}}{\partial x_{i}} + {\mathcal Q}_{rad} + {\mathcal Q}_{prec}.
\label{ptil_eq}
\end{equation}
In equation \ref{ptil_eq}, ${\mathbf \tau^{\theta}}$ is the subgrid scale contribution related to the potential temperature and ${\mathcal Q}_{rad}$ is a radiative source term 
(depending on the case this latter can be calculated via simplified formula or given by an atmospheric radiative transfer model). ${\mathcal Q}_{prec}$ is a precipitation 
related term which arises directly from the vertical redistribution of water vapor through hydrometeor precipitation. Following Wang and Chang \cite{WC1993} it can be written:
\begin{equation}
{\mathcal Q}_{prec} = \frac{\theta_{il}^{2}}{\theta}�\frac{L_{v}{\mathcal P}_{l} + L_{s}{\mathcal P}_{i}}{c_{p} T},
\end{equation}
with ${\mathcal P}_{l}$ and ${\mathcal P}_{i}$ the total precipitation fluxes of liquid and ice water. Expressions for these two fluxes are given below.
Let's recall that the ice/liquid potential temperature is related to the potential temperature through:
\begin{equation}
\theta_{il} = \theta \left(1 + \frac{L_{v}q_{l} + L_{s}q_{i}}{c_{p} T}\right)^{-1}.
\label{ptil}
\end{equation}

For a similar reason as for $\theta_{il}$, the total water mass $q_{t}$ is defined as a prognostic quantity in the model. This ensures that water mass is perfectly conserved 
throughout the simulation:
\begin{equation}
\frac{\partial q_{t}}{\partial t} + u_{i}\frac{\partial q{t}}{\partial u_{i}} = \frac{1}{\rho}\frac{\partial {\mathbf \tau^{qt}}}{\partial x_{i}} + {\mathcal S}_{prec}.
\label{qt_eq}
\end{equation}
Here again, ${\mathcal S}_{prec}$ denotes a source/sink of total water due to redistribution through precipitation.
The mass mixing ratio of water vapor is a diagnostic quantity which is evaluated by substracting the hydrometeors mixing ratios to $q_{t}$.

Finally, the other scalars, namely the mass mixing ratios and number concentrations of all the hydrometeors, are governed by the following equation:
\begin{equation}
\frac{\partial {\mathbf \Psi}}{\partial t} + u_{i}\frac{\partial {\mathbf \Psi}}{\partial x_{i}} - \frac{1}{\rho}\frac{\partial \rho\left<{\mathbf V_{p}}\right> 
{\mathbf \Psi}}{\partial z} = \frac{1}{\rho}\frac{\partial {\mathbf \tau^{\Psi}}}{\partial x_{i}} + {\mathbf S}_{micro}
\label{scalar}
\end{equation}
where ${\mathbf \Psi}=\left(q_{k},n_{k}\right)$, with k=1,....,$N_{h}$, $N_{h}$ being the total number of hydrometeors. In the above, 
${\mathbf S}_{micro}$ represents a microphysical source or sink per unit mass, ${\mathbf \tau^{\Psi}}$ is the subgrid scale contribution and 
$\left<{\mathbf V_{p}}\right>$ are the mass weighted precipitation velocities averaged over the hydrometeor populations (assumed to be $0$ for vapour and cloud droplets). 

All the thermodynamic quantities are related through the ideal gas law for a wet atmosphere:
\begin{equation}
P = \rho r_{d} T \left(1 + 0.61 q_{v}\right).
\end{equation}
This is used to retrieve the density and close the thermodynamic system. 

When the anelastic equations are solved (diagnostic pressure option), the equation system is closed by the incompressible constraint that the velocity divergence must 
vanish everywhere:
\begin{equation}
\frac{\partial \rho_{0} u_{i}}{\partial x_{i}} = 0.
\label{divergence}
\end{equation}
As explained in section \ref{pressure_solver}, the perturbation pressure is actually diagnostically computed at the end of each time step in order to make sure that this
constraint is respected.

\subsection{Subgrid scale closure}
\label{SGS}

SGS models are used to evaluate the subgrid scale contributions ${\mathbf \tau^{\Psi}}$ of all the scalars. By analogy with molecular diffusion, a simple hypothesis 
enables to write:
\begin{equation}
\tau_{i}^{\Psi} = -\rho K_{h} \frac{\partial \Psi}{\partial x_{i}}
\end{equation}
for the scalars and
\begin{equation}
\tau_{i,j}^{u} = -\rho K_{m} \left(\frac{\partial u_{i}}{\partial x_{j}} + \frac{\partial u_{j}}{\partial x_{i}}\right)
\end{equation}
for momentum. Two approaches can then be used to evaluate the eddy diffusivities $K_{h}$ and $K_{m}$.

The first approach is a 1st order closure based on the solution of a turbulent kinetic energy equation (Deardorff \cite{Dea1974}). A closed form of the SGS TKE 
equation can be written (without further details):
\begin{equation}
\frac{\partial e}{\partial t} + u_{i}\frac{\partial e}{\partial x_{i}} = 2K_{m}S_{ij}S_{ij} + \frac{2}{\rho}\frac{\partial}{\partial x_{i}}\left(\rho K_{m}
\frac{\partial e}{\partial x_{i}}\right) - K_{h} N^{2} - C_{\epsilon} \frac{e^{3/2}}{l}.
\label{tke}
\end{equation}
In the above equation, $e$ is the SGS TKE, $S_{ij} = \frac{1}{2}\left(\frac{\partial u_{i}}{\partial x_{j}} + \frac{\partial u_{j}}{\partial x_{i}}\right)$ is the 
strain tensor, $N^{2}$ is the Brunt-Vaisala frequency, and $l$ a characteristic length scale for the SGS processes. It is usually assumed that $K_{h} = 
K_{m}/Pr$, with $Pr$ the turbulent Prandtl number ($\approx 0.4$). The characteristic length scale is defined in the present LES model as (in 3D):
\begin{equation}
l = \left[\left(\Delta x \Delta y \Delta z\right)^{-2/3} + \left(z\kappa/C_{S}\right)^{-2}\right]^{-1/2},
\end{equation}
with $z$ the local altitude, $\kappa$ the Von Karman constant ($\approx 0.35$) and $C_{S}$ the Smagorinsky constant (defined below). 
The eddy diffusivity $K_{m}$ is finally obtained following scaling arguments:
\begin{equation}
K_{m} = C_{m} l \sqrt{e}.
\end{equation}
The constants $C_{m}$, $C_{\epsilon}$ and $C_{S}$ are given the values $0.15$, $0.93$ and $0.21$ respectively. 

The second closure is based on the 0th order Smagorinsky and Lilly approach \cite{Smago}. If we assume the balance between production of TKE by shear plus buoyancy and dissipation in 
equation \ref{tke}, it follows:
\begin{equation}
2K_{m}S_{ij}S_{ij} - K_{h} N^{2} - C_{\epsilon} \frac{e^{3/2}}{l} = 0. 
\end{equation}
Rearranging the terms yields:
\begin{equation}
K_{m} = \left(C_{S}l\right)^{2} \sqrt{2S_{ij}S_{ij}}\sqrt{1 - \frac{Ri}{Pr}},
\end{equation}
where the Richardson number $Ri = \frac{N^{2}}{\left|{\mathbf S}\right|}$ was introduced to account for convective stability. 

The Brunt-Vaisala frequency is modeled following Stevens et al. \cite{Sal2002} to include changes in stability related to condensation/evaporation at the SGS level.
$N^{2}$ is defined by:
\begin{equation}
K_{h}N^{2} = -\frac{g}{\theta_{00}}\left<w^{\prime}\theta_{v}^{\prime}\right> \approx K_{h}\frac{\partial b}{\partial z}.
\end{equation}
By considering the buoyancy production generated by the mixing of two separate levels in saturated or subsaturated conditions, Stevens et al. \cite{Sal2002} propose to model $N^{2}$
following:
\begin{equation}
N^{2} = \frac{g}{\theta_{00}}\left[\Delta \theta_{l} + 0.608 \theta_{00}\Delta q_{t} + \frac{n}{2}G\left(\Delta q_{t} - F \Delta�\theta_{l}\right)\right].
\end{equation}
Here, $n$ represents the number of saturated layer and the two parameters $F$ and $G$ are defined by : $F=\Pi d q_{s}/d T$ and 
$G = \left(L_{v}/c_{p}\Pi - \theta_{00}\left(1+r_{m}\right)\right)/\left(1 + L_{v}/c_{p} d q_{s}/d T\right)$. $n$ represents the number of consecutive saturated layers. In the model,
we only consider the two neighbouring layers at a given level so that $n$ is always set to 2 (for example within the cloud layer) or 0 (if the considered point is situated in subsaturated
conditions). The above relation was developed by considering that the cloud water in saturated conditions is adjusted to saturation. This model does not account for ice phase water
but we will still use it as such even when ice crystals are present.

The scalar eddy diffusivity is defined as the ratio between $K_{m}$ and a turbulent Prandtl number. In MIMICA, it is possible to set $Pr$ to a negative value, in
which case $K_{h}$ is assumed to decrease linearly up to an altitude of approximately 200 m over which $K_{h}$ remains close to 0. This assumption is equivalent to the
use of an implicit LES for scalars above the 200 m level where turbulent diffusion is only provided by numerical resolution and the dissipation of the numerical scheme.
Provided that all the scalars are advected using flux-limited schemes which generate numerical diffusion in regions with strong gradients (see section 
\ref{numerics}), it can be argued that numerically induced diffusion is already sufficient so that there is no need for additional diffusion through an explicit SGS contribution. 

One precision on the numerical discretization of turbulent diffusion terms must be added here. No particular procedure is applied to calculate the horizontal
diffusion: the gradients are computed by simple 2nd order central differences, which is made easier by the staggered grid system, and the explicit diffusive terms
are calculated at the same location as the velocity vector or scalars. However, due to the non-constant grid spacing in the vertical direction,
for instance close to the surface, the velocity and scalar gradients may generate numerical waves leading to instabilities. To avoid their development, 
vertical turbulent diffusion for both velocities and scalars is implicited: the vertical contributions are rewritten into a tridiagonal system which can easily be solved 
using standard methods. 

\subsection{Radiation}

Radiation in the LES model can be treated following to approaches: a detailed one making use of a separate radiation solver and a simplistic formulation where
the radiation fluxes are expressed as functions of LWP only (very suitable for LES intercomparison studies as in Stevens et al. \cite{Sal2005}). In {\it start}, the parameter
RADIA controls the choice of the model, if it's true the detailed radiation solver is used whereas in the other case, the simplified model is selected.

The radiation solver coupled to MIMICA is the 2005 version of the Fu-Liou model developed by Fu and Liou \cite{FL1993}. The solver is a multiband 2-4 stream radiation model and requires
certain input files and options to operate properly. The interface between MIMICA and the radiation model is provided by {\it radiag.f90} where the local data computed by the LES
are stored in 1D arrays in pressure dimension. These arrays constitute the input data required by {\it rad\_ir} and {\it rad\_vis} the two main routines calculating respectively the
longwave and shortwave radiative fluxes. The model requires the pressure array, the absolute temperature, the water vapor mixing ratio, the mass mixing ratios as well as the
effective radiuses of ice particles and cloud droplets, and finally ozone soundings that can be provided by the CHEMISTRY module. The solver provides in return the LW and SW upward 
and downward fluxes that have to be summed and derived to determine the net heating/cooling rates. Note that the McICA option, defined by default to FALSE in {\it radiag.f90}, controls
the multiband option of the solver: when turned on TRUE, the model solves for a single band instead of all the defined ones and extrapolates the results for the others. This allows to
considerably improve the overall code performances.

The radiation solver works on 1D columns extending up to the tropopause. Because the LES domain is limited to about 1500-2000 m in altitude, the input data arrays must be completed
using standard data above the LES domain. Standard atmospheres are stored in external .lay files located in the DATA directory (f.i. {\it kmlw.lay}). The most appropriate
standard atmosphere must then be copied in the local working directory and renamed {\it sounding\_rad.dat}. The user must be careful when using one of the predefined files
and must make sure that the standard and LES data match properly at the domain top (no sharp jump in temperature, no pressure inversion...). It is of course possible for
the user to build its own customized sounding file. In addition to the standard atmosphere, the radiation solver also requires two additional tables located in DATA containing 
informations used by the CKD (correlated k-distribution) model: {\it ckd.dat} and {\it cldwtr.dat}.

In addition to the 1D soundings extrapolated up to the tropopause, surface parameters must also be provided as inputs of the radiation solver. This includes the surface albedo and surface 
skin temperature, both prescribed in {\it cm.nml}, as well as the surface emissivity which is hard coded and set to $0.9$ in {\it radiag.f90}. To finish with, the solar zenith angle is 
computed at each radiation step at the beginning of {\it radiag.f90}. The formula uses the prescribed latitude, the julian day, the time reference for the simulation with respect to UTC 
and the instantaneous time in seconds. All the necessary parameters must be set in {\it cm.nml}.

The second simplistic method defines the net radiative fluxes as a function of LWP only:
\begin{equation}
F_{rad}\left(z\right) = F_{0}\exp\left[-k\left(LWP\left(z_{t}\right) - LWP\left(z\right)\right)\right] + F_{1}\exp\left[-k LWP\left(z\right)\right] + F_{3}\left(z\right),
\end{equation}
with $z_{t}$ the cloud top height, and
\begin{equation}
LWP\left(z\right) = \int_{0}^{z} \rho_{0}\left(z^{\prime}\right) q_{l}\left(z^{\prime}\right) dz^{\prime}.
\end{equation}
The coefficients $F_{0}$, $F_{1}$ and $k$ are case dependent and are usually fitted to detailed radiation calculations performed in similar conditions as the considered case.
$F_{3}\left(z\right)$ is an additional function of altitude sometimes present to compensate for cooling due to large scale subsidence above cloud top. 

Radiative heating tendencies are finally obtained following:
\begin{equation}
{\mathcal Q}_{rad} = -\frac{1}{\rho_{0}C_{p}} \frac{\partial F_{rad}}{\partial z}.
\end{equation}

\subsection{Microphysics: a two moment bulk scheme}
\label{micro}

\subsubsection{Basic definitions}

A two-moment bulk microphysics scheme is implemented in the code for all the hydrometeors (including cloud droplets) where assumed size distributions are used, taking the
form of regular gamma functions denoted $\Gamma\left(x\right)$. They are expressed as:
\begin{equation}
dN = N_{0} D^{\alpha} e^{-\lambda D} dD,
\label{distribution}
\end{equation}
where $N$ is the number concentration of a given hydrometeor population having diameter $D$, $\alpha$ is a constant depending on the hydrometeor type, and $N_{0}$ and
$\lambda$ are two parameters which can be expressed in terms of $N$ and $Q$ the mass mixing ratio. 
\begin{equation}
N_{0} = \frac{N\lambda^{\alpha+1}}{\Gamma\left(\alpha+1\right)} \mbox{\hspace{2cm} and \hspace{2cm}} 
\lambda = \left(\frac{A_{m}\Gamma\left(\alpha+\beta_{m}+1\right)N}{\Gamma\left(\alpha+1\right)Q}\right)^{1/\beta_{m}}.
\end{equation}
The parameters $A_{m}$ and $\beta_{m}$ are the constants in the particle's mass-size relation given by:
\begin{equation}
m = A_{m} D^{\beta_{m}}.
\label{mass}
\end{equation}
One important property of $\Gamma$ that will be used later on must be introduced here: $\Gamma\left(x+1\right) = x\Gamma\left(x\right)$.

$N$ and $Q$ are defined by:
\begin{equation}
N = \int_{0}^{+\infty} f\left(D\right) dD \mbox{\hspace{2cm} and \hspace{2cm}} Q = \int_{0}^{+\infty} m f\left(D\right) dD,
\label{N_et_Q}
\end{equation}
with $f\left(D\right) = \frac{dN}{dD}$.
Therefore, $N$ is considered to be the 0th moment of distribution \ref{distribution} while $m$ is the $\beta_{m}$th moment. The mean diameter of a given hydrometeor 
population, which will be used later, is the first moment of \ref{distribution}: $\overline{D} = \int_{0}^{+\infty} D f\left(D\right) dD$. 
The values of coefficients $\alpha$, $\beta_{m}$ and $A_{m}$ for all the hydrometeor kinds defined in the model are summarized in table \ref{micro_param}. 
Note that species terminal fall speeds, $V_{p}$, are modeled using simple power laws of the diameter as well (the coefficients are gathered in table \ref{micro_param}):
\begin{equation}
V_{p} = A_{v} D^{\beta_{v}} \left(\frac{\rho_{0}}{\rho}\right)^{\gamma}.
\label{TFS}
\end{equation}
Cloud droplets are commonly assumed not to precipitate so that $A_{v,c}=0$.

At the moment, five types of hydrometeors are described by the model: cloud droplets, rain drops, cloud ice, graupel and snow. In the following, quantities refering to a specific
type of hydrometeor will be denoted by the initials c, r, i, g or s. 

Ice crystals may take different shapes depending on the thermodynamic conditions under which they were nucleated. At the moment, only one type of ice crystal habit can be assumed 
at a time for each simulation. The proper ice habit has to be selected prior to the beginning of the calculation in {\it shared\_hydro.f90} where a list of ice microphysical 
parameters is provided (hard coded). All the default ice crystal types are summarized in table \ref{micro_param}. 

\begin{table}[!h]
\begin{center}
\begin{tabular}{c|ccccc}
hydrometeor & $\alpha$ & $\beta{m}$ & $A_{m}$ & $\beta_{v}$ & $A_{v}$ \\
\hline
cloud (c) & 5 & 3 & 523.6 & - & - \\
rain (r) & 0 & 3 & 523.6  & 0.8 & 836. \\
ice (i, rosette C2a) & 2 & 2.26 & 0.102 & 1.225 & 6059.5 \\
ice (i, columns C1f) & 2 & 1.8 & 0.0093 & 0.488 & 30.14 \\
ice (i, plates P1c) & 2 & 2.79 & 1.43 & 0.62 & 17.9 \\
ice (i, dendrites P1e) & 2 & 2.29 & 0.0233 & 0.48 & 5.02 \\
ice (i, assemblage of den. P7b) & 2 & 2.68 & 0.35 & 0.83 & 62.62 \\
graupel (g) & 0 & 3 & 65  & 0.8 & 199.05 \\
snow (s) & 1 & 2 & 0.04   & 0.333 & 6.962 \\
\end{tabular}
\end{center}
\caption{{\it Coefficients for the microphysics scheme for all five kinds of hydrometeors. Parameters for ice crystals are taken from Propacher and Klett \cite{PK1997} and Khvorostyanov 
and Curry \cite{KC2002}. Units are SI (kg-m-s).}}
\label{micro_param}
\end{table}

The choice of the microphysics level is controled by the integer lmicro in {\it cm.nml}: lmicro = 0 corresponds to a dry case where no condensed water is present (no
microphysics calculated), for lmicro = 1, only warm microphysics is modeled (cloud drops and rain, drizz in {\it cm.nml} even allows to swith on and off the formation of
precipitating rain drops), for lmicro = 2, ice microphysics is included but only for pristine ice crystals (no riming and no aggregation) whereas lmicro = 3
includes the formation of snow flakes and graupel (the 5 hydrometeor species, $nh=5$).

$nh$ and $nhydro$ must not be confused: $nhydro$ is the total possible number of hydrometeors which is used to allocate and initialize all the arrays depending on
microphysics, whereas $nh$ is the number of modeled hydrometeors and is used for the calculations. 

\subsubsection{Equivalence with mass distributions}

It is sometimes convenient to describe hydrometeor populations in terms of mass distributions instead of size. Combining relations \ref{mass} and \ref{distribution} yields 
a distribution under the form of a generalized gamma distribution:
\begin{equation}
dN = M_{0} m^{\nu} e^{-\Lambda m^{\mu}} dm,
\end{equation}
where $M_{0}$ and $\Lambda$ are the new parameters that need to be redefined, and $\nu$ and $\mu$ two constants dependent on the type of hydrometeor. Comparing the mass
and size distributions the following equalities come:
\begin{eqnarray}
M_{0} &=& \frac{N_{0}}{\beta A_{m}^{\frac{\alpha+1}{\beta_{m}}}}, \\
\Lambda &=& \frac{\lambda}{A_{m}^{1/\beta_{m}}}, \\
\nu &=& \frac{\alpha +1}{\beta_{m}} - 1, \\
\mu &=& \frac{1}{\beta_{m}}.
\end{eqnarray}

The kth moment of this generalized distribution is given by:
\begin{eqnarray}
M^{\left(k\right)} &=& \int_{0}^{+\infty} m^{k} f\left(m\right) dm \label{relation1}\\
 &=& M_{0}\int_{0}^{+\infty} m^{k+\nu} e^{-\Lambda m^{\mu}} dm \\
 &=& M_{0} \frac{\Gamma\left(\frac{\nu+k+1}{\mu}\right)}{\mu} \Lambda^{-\frac{\nu+k+1}{\mu}} \label{relation3}.
\end{eqnarray}
Note that using relation \ref{relation3}, expressions for $M_{0}$ and $\Lambda$ as functions only of $N$, $Q$ and the parameters $\nu$ and $\mu$ can be derived:
\begin{equation}
M_{0} = \frac{\mu N}{\Gamma\left(\frac{\nu+1}{\mu}\right)} \Lambda^{\frac{\nu+1}{\mu}} \mbox{\hspace{2cm} and \hspace{2cm}}
\Lambda = \left[\frac{\Gamma\left(\frac{\nu+1}{\mu}\right) Q}{\Gamma\left(\frac{\nu+2}{\mu}\right) N}\right]^{-\mu}.
\end{equation}
Replacing the $\nu$ and $\mu$ by their corresponding expressions depending on $\alpha$, $A_{m}$ and $\beta_{m}$, we find:
\begin{equation}
M_{0} = \frac{N}{\beta\Gamma\left(\alpha+1\right)} \Lambda^{\alpha+1} \mbox{\hspace{2cm} and \hspace{2cm}}
\Lambda = \left[\frac{\Gamma\left(\alpha+1\right) Q}{\Gamma\left(\alpha+\beta+1\right) N}\right]^{-1/\beta}.
\end{equation}

\subsubsection{Collisions between hydrometeors}

The collisional sources and sinks are obtained by integrating a collection kernel $K\left(m,m^{*}\right)$ over two different hydrometeor populations having masses $m$
and $m^{*}$. The time variation of a mass distribution $f\left(m\right)$ due to collision processes can be derived, commonly called the Stochastic Collection Equation:
\begin{equation}
\frac{\partial f\left(m\right)}{\partial t} = \frac{1}{2}\int_{0}^{m} f\left(m-m^{*}\right) f\left(m^{*}\right) K\left(m-m^{*},m^{*}\right)dm^{*} -
\int_{0}^{+\infty}f\left(m\right)f\left(m^{*}\right)K\left(m,m^{*}\right)dm^{*}.
\end{equation}
Introducing the kth moment $M^{\left(k\right)}$ of $m$ in the previous relation, an equation for the evolution of
$M^{\left(k\right)}$ can be found \cite{SB2001}. From this general equation (not shown here), and after further simplifications, some simpler expressions can be derived giving 
the rate of change of the particle's number densities and mass mixing ratios due to collision/coalescence events. 
\begin{equation}
\frac{\partial N}{\partial t} = \int_{0}^{+\infty}f\left(m\right)\left[\int_{0}^{+\infty}f\left(m^{*}\right)K\left(m,m^{*}\right)dm^{*}\right] dm,
\label{number_int}
\end{equation}
and: 
\begin{equation}
\frac{\partial Q}{\partial t} = \int_{0}^{+\infty}mf\left(m\right)\left[\int_{0}^{+\infty}f\left(m^{*}\right)K\left(m,m^{*}\right)dm^{*}\right] dm,
\label{mass_int}
\end{equation}
The previous equations represent the evolution of the number and mass of a given hydrometeor species $\Psi$ due to collisions with particles $\Psi^{*}$ belonging to another species. 
These relations can be derived without any assumption on the collection kernel or on the mass distribution. 

Given that $f$ is usually assumed to take the shape of gamma distribution, the only unknown in the previous equations is the collection kernel $K$ which needs to be presumed. 
The original model uses the gravitational kernel which reads:
\begin{equation}
K\left(D,D^{*}\right) = \rho E^{*}\frac{\pi}{4}\left(D+D^{*}\right)^{2}\left|V_{p} - V_{p}^{*}\right|.
\label{grav_ker}
\end{equation}
The coefficient $E^{*}$ represents the collision efficiency and usually depends on the considered microphysical species and local conditions. 
This modelling assumption is still in use in the mixed-phase microphysics part of the LES. Warm microphysics processes are however now modeled using 
Seifert and Beheng's approach which makes use of a slightly different kernel as shown in the next section.

Provided that the SCE is given in terms of particle masses, it may not be adequate to represent an hydrometeor population by its size distribution. This issue is however commonly
overcome by simply assuming that the previous relations for the mass and number microphysical sources can directly be extended to a size distribution. The correspondance between
the two relations may not be perfectly exact as mass and diameter are related by a linear power law formula in such a way that additional terms or factors should be included.
The assumption seems nevertheless acceptable and will therefore be used in the following.

\subsubsection{Warm microphysics: Seifert \& Beheng's scheme}

Warm microphysics (involving only liquid water) will be treated using Seifert and Beheng's model \cite{SB2001, SB2004}. The microphysics parameterization uses a double moment
approach for both cloud droplets and rain where mass distributions of these two species are represented by generalized gamma functions. The approach chosen by Seifert and
Beheng is thus slightly different from the framework introduced previously, but the same assumptions still apply.

The authors use a simple kernel defined by a picewise polynome (Long's kernel):
\begin{eqnarray}
K\left(m,m^{*}\right) &=& k_{c}\left(m^{2} + m^{*2}\right) \\
K\left(m,m^{*}\right) &=& k_{r}\left(m + m^{*}\right).
\end{eqnarray}
The first polynome is used when the mass of the collector $m$ is smaller than a critical mass $m_{c}$ separating cloud drops and rain drops. This mass is calculated
from the separation diameter between cloud droplets and rain prescribed by the parameter {\it drmin} in {\it cm.nml}. Cloud drops can only
collect smaller cloud drops so that the first polynome will only represent cloud/cloud interactions. The second one is used when the collector's mass is greater than
$m_{c}$, which corresponds to rain/cloud or rain/rain interactions. The constants $k_{c}$ and $k_{r}$ are given the values $9.44e9$ cm3/g2/s and $5.78e3$ cm3/g/s
respectively. 

Integrating the mass distributions following relations \ref{number_int}, \label{mass_int} and using the above kernels leads to the exact expressions for the rates of 
autoconversion (c+c $\rightarrow$ r), accretion (r+c $\rightarrow$ r) and selfcollection (c+c $\rightarrow$ c or r+r $\rightarrow$ r). These rates can be found in 
\cite{SB2001, SB2004} and are recalled hereafter:
\begin{equation}
\left.\frac{\partial Q_{r}}{\partial t}\right|_{auto} = auq = \frac{k_{c}}{20 m_{c}} \frac{\left(\nu_{c}+2\right)\left(\nu_{c}+4\right)}{\left(\nu_{c}+1\right)
\left(\nu_{c}+1\right)} Q_{c}^{2} \overline{m}_{c} \left[1+\frac{\Phi_{a}\left(\tau\right)}{\left(1-\tau\right)^{2}}\right],
\label{auq}
\end{equation}
\begin{equation}
\left.\frac{\partial N_{r}}{\partial t}\right|_{auto} = aun = \frac{1}{m_{c}} \left.\frac{\partial Q_{r}}{\partial t}\right|_{auto}
\label{aun}
\end{equation}
\begin{equation}
\left.\frac{\partial Q_{c}}{\partial t}\right|_{auto} = -auq 
\label{ccq}
\end{equation}
\begin{equation}
\left.\frac{\partial N_{c}}{\partial t}\right|_{auto} = -2\times aun
\label{ccn}
\end{equation}
\begin{equation}
\left.\frac{\partial Q_{r}}{\partial t}\right|_{acc} = clr = k_{r} Q_{c} Q_{r} \Phi_{ac}\left(\tau\right)
\label{crrq}
\end{equation}
\begin{equation}
\left.\frac{\partial N_{r}}{\partial t}\right|_{acc} = 0
\label{crrn}
\end{equation}
\begin{equation}
\left.\frac{\partial Q_{c}}{\partial t}\right|_{acc} = -clr
\label{crcq}
\end{equation}
\begin{equation}
\left.\frac{\partial N_{c}}{\partial t}\right|_{acc} = clrn = -\frac{1}{\overline{m}_{c}} clr
\label{crcn}
\end{equation}
\begin{equation}
\left.\frac{\partial Q_{r}}{\partial t}\right|_{sc} = 0
\label{scrq}
\end{equation}
\begin{equation}
\left.\frac{\partial N_{r}}{\partial t}\right|_{sc} = crrn = -k_{r} N_{r} Q_{r} \Phi_{b}\left(\tau\right) 
\label{scrn}
\end{equation}
\begin{equation}
\left.\frac{\partial Q_{c}}{\partial t}\right|_{sc} = 0
\label{sccq}
\end{equation}
\begin{equation}
\left.\frac{\partial N_{c}}{\partial t}\right|_{sc} = cccn = -k_{c}\frac{\left(\nu_{c}+2\right)}{\left(\nu_{c}+1\right)}Q_{c}^{2} + 2\times aun 
\label{sccn}
\end{equation}
where $\nu_{c}$ corresponds to the mass exponent in the mass distribution for cloud droplets. Seifert and Beheng suggest that this parameter should be set to $1$, which is
equivalent to $\alpha = -1/3$ in a size distribution (considering that $\beta_{m}=3$). To remain consistent with table \ref{micro_param} where $\alpha = 5$, a value of 
$\nu_{c}=0$ will be chosen (set in {\it shared\_data}). 

In the above, $\overline{m} = Q/N$ is a mean mass, $\tau=1-\frac{Q_{c}}{Q_{c}+Q_{r}}$ is a dimensionless scaling parameter, and $\Phi_{a}$ and $\Phi_{b}$ are autoconversion,
accretion and break-up rates defined by Seifert and Beheng \cite{SB2004} using similarity assumptions as:
\begin{equation}
\Phi_{a} = 400\tau^{0.7}\left(1-\tau^{0.7}\right)^{3}
\end{equation} 
\begin{equation}
\Phi_{ac} = \left(\frac{\tau}{\tau + 5e-5}\right)^{4}
\end{equation} 
\begin{eqnarray}
\Phi_{b} = k_{b}\left(\overline{D}_{r} - D_{eq}\right) &\mbox{\hspace{1cm} when \hspace{1cm}}& 0.35 mm < \overline{D}_{r} < D_{eq} \\
\Phi_{b} = 2\exp\left[k_{b}\left(\overline{D}_{r} - D_{eq}\right)\right] - 1 &\mbox{\hspace{1cm} when \hspace{1cm}}& \overline{D}_{r} > D_{eq}
\end{eqnarray} 
with $k_{b}=1000$ 1/m and $D_{eq}$ an equilibrium diameter (corresponding to a balance between break-up and selfcollection) set to $0.9$ mm. $\Phi_{b}$ is set to $0$ when
$\overline{D}_{r}<0.35$ mm. 

\subsubsection{mixed phase microphysics}

Mixed-phase microphysics is treated using a similar parameterization as in the original CRM model. Three ice microphysical species are described (ice, graupel and snow)
and modeled via their size distributions. The gravitational kernel \ref{grav_ker} is used to represent their interactions with any other hydrometeor (including liquid
drops). 

The collision rates between two hydrometeors, 1 collecting 2, for cold microphysics are given by:
\begin{eqnarray}
P_{n12} &=& \frac{\pi}{4} \rho E_{12} \int_{0}^{+\infty}\int_{0}^{+\infty}f\left(D_{1}\right)f\left(D_{2}\right) \left(D_{1} + 
D_{2}\right)^{2} \left|V_{p1} - V_{p2}\right|dD_{1} dD_{2},
\end{eqnarray}
\begin{eqnarray}
P_{q12} &=& \frac{\pi}{4} \rho E_{12} \int_{0}^{+\infty}\int_{0}^{+\infty}D_{2}f\left(D_{1}\right)f\left(D_{2}\right) \left(D_{1}
+ D_{2}\right)^{2} \left|V_{p1} - V_{p2}\right|dD_{1} dD_{2},
\end{eqnarray}
After tidious manipulations and simplifications, one can write the general form of the collision rates:
\begin{eqnarray}
\begin{split}
P_{n12} = \frac{\pi}{4} \rho E_{12} N_{1}N_{2} \left[ \left<D_{1}^{2}\right> \left|\frac{\left(\alpha_{1}+\beta_{v1}+2\right)\left(\alpha_{1}+\beta_{v1}+1\right)}
{\left(\alpha_{1}+2\right)\left(\alpha_{1}+1\right)}\left<V_{p1}\right> - \left<V_{p2}\right>\right|\right. \\
+2\left<D_{1}\right>\left<D_{2}\right>\left|\frac{\alpha_{1}+\beta_{v1}+1}{\alpha_{1}+1}\left<V_{p1}\right> - \frac{\alpha_{2}+\beta_{v2}+1}{\alpha_{2}+1}
\left<V_{p2}\right>\right| \\
\left.+ \left<D_{2}^{2}\right> \left|\left<V_{p1}\right> - \frac{\left(\alpha_{2}+\beta_{v2}+2\right)\left(\alpha_{2}+\beta_{v2}+1\right)}
{\left(\alpha_{2}+2\right)\left(\alpha_{2}+1\right)}\left<V_{p2}\right>\right|\right]
\label{collision_n}
\end{split}
\end{eqnarray}

\begin{eqnarray}
\begin{split}
P_{q12} = \frac{\pi}{4} \rho E_{12} Q_{1}N_{2} \left[ \left<D_{1}^{2}\right>_{m} \left|\frac{\left(\alpha_{1}+\beta_{v1}+\beta_{m1}+2\right)\left(\alpha_{1}+\beta_{v1}+\beta_{m1}+1\right)}
{\left(\alpha_{1}+\beta_{m1}+2\right)\left(\alpha_{1}+\beta_{m1}+1\right)}\left<V_{p1}\right>_{m} - \left<V_{p2}\right>\right|\right. \\
+2\left<D_{1}\right>_{m}\left<D_{2}\right>\left|\frac{\alpha_{1}+\beta_{v1}+\beta_{m1}+1}{\alpha_{1}+\beta_{m1}+1}\left<V_{p1}\right>_{m} - \frac{\alpha_{2}+\beta_{v2}+1}{\alpha_{2}+1}
\left<V_{p2}\right>\right| \\
\left. + \left<D_{2}^{2}\right> \left|\left<V_{p1}\right>_{m} - \frac{\left(\alpha_{2}+\beta_{v2}+2\right)\left(\alpha_{2}+\beta_{v2}+1\right)}
{\left(\alpha_{2}+2\right)\left(\alpha_{2}+1\right)}\left<V_{p2}\right>\right|\right]
\label{collision_q}
\end{split}
\end{eqnarray}
The notations $\left<.\right>$ and $\left<.\right>_{m}$ respectively denote an ensemble mean over the entire hydrometeor population and a mass weighted ensemble mean.
Elementary functions were created in MIMICA to calculate the mean diameters and terminal fall speeds as well as any other useful distribution averaged qantity. $P_{q12}$ represents 
the mass tendency of species 1 when collected by species 2. It is also possible in a similar manner to define the mass tendency of species 2 following the same process.

Two major simplifications can be obtained in the case where two identical hydrometeors interact (selfcollection) or when the collected
particle is a cloud droplet (riming). In the latter case, two assumptions can be made: that the cloud droplet terminal fall speed is 0, and that their size or mass is small
compared to the collector particle. We can then rewrite $P_{n11}$ as well as $P_{n1c}$ under the following forms (similar simplifications are obtained for $P_{q}$):
\begin{eqnarray}
\begin{split}
P_{n11} = \frac{\pi}{2} \rho E_{11} N_{1}^{2} \left<D_{1}^{2}\right> \left<V_{1}\right>\left|\frac{\left(\alpha_{1}+\beta_{v1}+2\right)\left(\alpha_{1}+\beta_{v1}+1\right)}
{\left(\alpha_{1}+2\right)\left(\alpha_{1}+1\right)} - 1\right|,
\label{coll_same}
\end{split}
\end{eqnarray}
(during selfcollection, the mass of 1 is conserved in such a way that $P_{q11}=0$)
\begin{equation}
\begin{split}
P_{n1c} = \frac{\pi}{4} \rho E_{1c} N_{1}N_{c}\frac{\left(\alpha_{1}+\beta_{v1}+2\right)\left(\alpha_{1}+\beta_{v1}+1\right)}
{\left(\alpha_{1}+2\right)\left(\alpha_{1}+1\right)}\left<D_{1}^{2}\right>\left<V_{p1}\right>.
\label{coll_drop}
\end{split}
\end{equation}
These general relations can be applied to any type of hydrometeor provided that the coefficients for the mass and terminal fall speed formula are known.

For most of the interactions, the preceding relations are applied as such. All the possible interactions treated in the code are summarized in table \ref{collisions}.
A total of 17 processes are modeled in the present version. Some collision processes present particularities. Riming of ice crystals and snow
(resulting from the collision between ice/snow and cloud droplets) is for instance assumed to occur only for cloud droplets having a diameter larger than 15 mm (see 
\cite{WC1993}). In these cases, a partial mixing ratio of cloud droplets effectively rimmed is calculated and multiplied to the integral collection rates. This allows
to exclude part of the cloud droplet spectrum from the rimming process. Similarly, autoconversion of ice particles to graupel will occur if the ice particle diameter 
exceeds a threshold $\hat{D}_{i} = 0.3$ mm. This transition can only result from deposition growth. In that case, only partial integrals must be calculated (see \cite{WC1993}).  

%\begin{landscape}
\begin{table}[!h]
\begin{center}
\begin{tabular}{|c|ccccc|}
\hline
hydrometeor types & cloud drops & rain & ice crystals & graupel & snow \\
\hline
            & c + c $\rightarrow$ c/r & c + r $\rightarrow$ r &  c + i $\rightarrow$ g & c + g $\rightarrow$ g & c + s $\rightarrow$ s \\
cloud drops & auq, aun, cccn                &  clr, clrn  & cli, clii, clin & clg, clgn & cls, clsn  \\
            & \ref{auq}, \ref{aun}, \ref{sccq}, \ref{sccn} & \ref{crcn}, \ref{crcq} & \ref{coll_drop} & \ref{coll_drop}  & \ref{coll_drop} \\
\hline
     &  & r + r $\rightarrow$ r &  r + i $\rightarrow$ g & r + g $\rightarrow$ g & r + s $\rightarrow$ g \\ 
rain &  & crrn (inc. break-up) & cir, cirn & crg, crgn & csr, csrr, csrn \\
     &  & \ref{scrn} & \ref{collision_n}, \ref{collision_q} & \ref{collision_n}, \ref{collision_q} & \ref{collision_n}, \ref{collision_q} \\
\hline
     &       &  & i + i $\rightarrow$ s & i + g $\rightarrow$ g & i + s $\rightarrow$ s \\ 
ice crystals &  &  & i $\rightarrow$ g (autoconv.) & - & cis, cisn \\
     &       &  & ais, aisn \ref{coll_same}, aig, aign & efficiency = 0 & \ref{collision_n}, \ref{collision_q} \\
\hline
        &  &  &  & g + g $\rightarrow$ g & g + s $\rightarrow$ g \\ 
graupel &  &  &  & - & csg, csgn \\
        &  &  &  & rebound & \ref{collision_n}, \ref{collision_q} \\
\hline
        &  &  &  &  & s + s $\rightarrow$ s \\ 
snow &  &  &  &  & cssn \\
        &  &  &  &  & \ref{coll_same} \\
\hline
\end{tabular}
\end{center}
\caption{{\it Interactions between hydrometeors treated in the code.}}
\label{collisions}
\end{table}
%\end{landscape}

\subsubsection{Precipitation}

Precipitation fluxes appearing in equation \ref{scalar} include a mean terminal fall speed for each hydrometeor kind, defined as a mass weighted average over the entier 
hydrometeor population. The power law \ref{TFS} defines the terminal fall speed of an hydrometeor having a given mass $m$, including a correction term accounting for 
vertical density variations. The mass weighted mean terminal fall speeds are thus defined as:
\begin{equation}
\left<V_{p}\right> = \frac{\int_{0}^{+\infty} m V_{p}\left(D\right) f\left(D\right) dD}{\int_{0}^{+\infty} m f\left(D\right) dD}.
\end{equation}
After simplifications, it comes:
\begin{equation}
\left<V_{p}\right> = A_{v} \frac{\Gamma\left(\alpha+\beta_{v}+\beta_{m}+1\right)}{\Gamma\left(\alpha+\beta_{m}+1\right)} \lambda^{-\beta_{v}} \left(\frac{\rho_{0}}{\rho}\right)^{\gamma}.
\end{equation}

\subsubsection{Phase transitions}
\label{phasetrans}

Here is defined how phase transitions are treated in the model: evaporation/condensation, sublimation/deposition, melting/freezing. Following Pruppacher and Klett
\cite{PK1997}, the mass tendency of a given droplet of diameter $D$ due to evaporation/condensation is given by:
\begin{equation}
\left.\frac{\partial m}{\partial t}\right|_{e/c} = 2\pi D G_{lv}\left(T,p\right) F_{v} \frac{s}{q_{s}},
\label{evap}
\end{equation}
with:
\begin{equation}
G_{lv}\left(T,p\right) = \left[ \frac{R_{w} T}{e_{s} D_{v}} + \frac{L_{lv}}{K_{T} T}\left(\frac{L_{lv}}{R_{w} T}-1\right)\right]^{-1}
\end{equation}
$s = q_{v} - q_{s}$ is the absolute supersaturation, $q_{s}$ the saturation mixing ratio over liquid water, $e_{s}$ the saturation vapor pressure over liquid water, $R_{w}$ 
the ideal gas specific constant for water vapor, $D_{v}$ the diffusion coefficient for water vapor ($\approx 3.e-5$ m2/s), $K_{T}$ the heat conductivity ($\approx 2.5e-2$ Jm/s/K).
The ventilation factor $F_{v}$ is defined following Chen and Lamb \cite{CL1994} as a function of the Best number $X = Sc^{1/3}Re^{1/2}$: 
\begin{equation}
F_{v} = a_{v} + b_{v} X^{\gamma}.
\end{equation}
The Schmidt number $Sc$ is assumed to be constant and equal to $0.7$ whereas the Reynolds number $Re$ is defined by:
\begin{equation}
Re = \frac{\rho_{0} V_{p} D}{\mu},
\end{equation}
with $V_{p}$ the drop terminal fall speed and $\mu$ the dynamic viscosity of air ($\approx 1.5e-5$ kg/m/s). For $ X \leq 1$, Chen and Lamb suggest that $\gamma = 2$, $a_{v}=1$ 
and $b_{v}=0.14$ whereas when $X>1$, $\gamma = 1$, $a_{v}=0.86$ and $b_{v}=0.28$ for ice crystals and $\gamma = 1$, $a_{v}=0.78$ and $b_{v}=0.308$ otherwise. Ventilation effects 
are neglected for cloud droplets so that $F_{v}$ is set to $1$. For the other hydrometeor species, it is possible to turn on the ventilation effects by setting lvent to 1 in 
{\it cm.nml} (these effects are not included by default but might have a large impact on ice crystal growth).

In order to obtain condensation/evaporation sources for the entire drop population, one has to integer relation \ref{evap} over the assumed size distribution.
To simplify integration, we assume that the final total mass tendency will take the form:
\begin{eqnarray}
\left.\frac{\partial Q}{\partial t}\right|_{e/c} &=& 2 \pi G_{lv}\left(T,p\right) \frac{s}{q_{s}} \int_{0}^{+\infty} D F_{v} f\left(m\right) dm \\
 &=& 2 \pi N G_{lv}\left(T,p\right) \left<D\right> F_{v}^{*} \frac{s}{q_{s}}
\label{cond/evap}
\end{eqnarray}
where we defined a modified ventilation factor depending on an averaged Reynolds number:
\begin{eqnarray}
F_{v}^{*} &=& a_{v} + b_{v}Sc^{1/3}\left<Re\right>^{1/2} \\
 &=&  a_{v} + b^{*}_{v}Sc^{1/3}\left(\frac{\beta_{v} + \alpha + 1}{\alpha + 1} \frac{\left<V\right>\left<D\right>}{\nu}\right)^{1/2}.
\end{eqnarray}
The mass weighted terminal fall speeds are defined previously and $b^{*}_{v}$ is a modified factor, not detailed here.

Similar relations are found for sublimation/deposition processes applied to ice crystals, graupel and snow flakes:
\begin{equation}
\left.\frac{\partial m}{\partial t}\right|_{e/c} = 4\pi C_{i} D G_{iv}\left(T,p\right) F_{v} \frac{s_{i}}{q_{si}},
\label{sublim}
\end{equation}
and:
\begin{equation}
G_{iv}\left(T,p\right) = \left[ \frac{R_{w} T}{e_{si} D_{v}} + \frac{L_{iv}}{K_{T}T}\left(\frac{L_{iv}}{R_{w} T}-1\right)\right]^{-1},
\end{equation}
which yields the following total mass tendencies for the solid hydrometeor populations:
\begin{equation}
\left.\frac{\partial Q}{\partial t}\right|_{s/d} = 4\pi N C_{i} G_{iv}\left(T,p\right) \left<D\right> F_{v}^{*} \frac{s_{i}}{q_{si}},
\end{equation}
$C_{i}$ is the ice crystal capacitance. It takes the value $1/2$ for spherical particles (graupel) and $1/\pi$ for other shapes. The assumption of a constant capacitance
is actually very approximate and more advanced methods with a capacitance depending on crystals shape (through their aspect ratio for example) are more accurate. 

The integration of condensation/evaporation sources for water droplets may sometimes lead to numerical instability issues if no specific treatment
is applied. Most of the time, these problems lead to the apparition of spurious water mixing ratio peaks at the cloud boundaries where in-cloud air parcels are mixed with the
ambient dry atmosphere. Direct integration of the condensation/evaporation sources are nonetheless necessary in two particular cases (even though simpler approaches can be 
used but with a dramatic drop in model accuracy): when three-phase microphysics is used so that water vapor will interact both with liquid drops and ice crystals, and when
droplet activation is modeled, especially with prognostic aerosols. These issues and the solution adopted in MIMICA are discussed in the following sections.

Melting of ice particles is treated in a similar manner as diffusional growth, with the mass tendency of a single particle given by:
\begin{equation}
\left.\frac{\partial m}{\partial t}\right|_{melt} = \frac{2\pi}{L_{il}} \left[K_{T}\left(T-T_{0}\right)F_{h} + \frac{D_{v}}{L_{lv}}\left(\frac{p_{v}}{T} -
\frac{p_{s}}{T_{0}}\right) F_{v}\right] D.
\end{equation}
$F_{h}$ is a heat ventilation coefficient approximated by $F_{h} = Le_{v} F_{v}$ with $Le_{v} \approx 0.67$ the Lewis number for water vapor, and $T_{0}=273.15$ K is
the freezing temperature. Integrating over the particle population gives:
\begin{equation}
\left.\frac{\partial Q}{\partial t}\right|_{melt} = \frac{2\pi}{L_{il}} N \left[K_{T}\left(T-T_{0}\right)Le_{v} + \frac{D_{v}}{L_{lv}}\left(\frac{p_{v}}{T} -
\frac{p_{s}}{T_{0}}\right)\right] F_{v}^{*} \left<D\right>,
\end{equation}
with $F_{v}^{*}$ defined as previously. All the melted water is considered to be rain.

In the above, positive tendencies mean growth of drops or ice particles via condensation or deposition of water vapor. Negative tendencies mean a loss of ice or liquid water 
via sublimation or evaporation. Whereas in the first case the number concentration of hydrometeors is left unchanged (the existing particles only grow), 
evaporation and sublimation lead to a decrease in droplet and ice crystal number concentrations. 
These variations of number concentration for ice particles and liquid drops due to sublimation and evaporation are expressed following:
\begin{equation}
\left.\frac{\partial N}{\partial t}\right|_{e/s} = \frac{N}{Q}\min\left(\left.\frac{\partial Q}{\partial t}\right|_{e/s}, 0\right).
\end{equation}
Here, the subscripts $e/s$ mean evaporation or sublimation as we consider both cloud droplets and ice crystals. The mass mixing ratio tendencies on the right
hand side are limited by $0$ as number concentrations are only allowed to decrease through evaporation/sublimation. 

\subsubsection{Saturation adjustment}

When a detailed knowledge of supersaturation is not required, the saturation adjustment method can be employed, thus overcoming the need to
integrate condensation/evaporation sources. The saturation adjustment assumption comes from the observation that in most stratiform clouds, in-cloud water
relaxes very quickly to equilibrium. In other words, omitting CCN activation, it can be assumed that at any time, the in-cloud atmosphere is saturated, $q_{v} = q_{s}$ or $s=0$,
so that all the water vapor available above saturation condenses instantaneously on the existing drops. Striclty following this assumption, the liquid water mixing ratio can  
be diagnosed given $q_{v}$ and $q_{s}$. The method is selected by setting the option SAT\_ADJ to TRUE in {\it start}.

Let's note $q_{l}^{n}$ and $q_{v}^{n}$ the quantities at time $t^{n}$, before adjustment, and $q_{l}^{*}$ and $q_{v}^{*}$ the same quantities after adjustment. 
Knowing that $q_{t} = q_{v} + q_{l}$, the total  water content, is conserved, we can write:
\begin{equation}
q_{l}^{*} = q_{t}^{n} - q_{s}\left(P,T^{*}\right).
\label{satadj1}
\end{equation}
Here, $q_{s}$ is expressed following Rogers and Yau \cite{RY1989}:
\begin{equation}
q_{s} = \frac{380.1}{P}\exp\left(\frac{T - 273.15}{T - 29.65}\right).
\end{equation}
The liquid water mixing ratio can be introduced in the previous relation by introducing the liquid potential temperature: $T = \Pi \Theta_{l} + \frac{L_{lv}q_{l}}{c_{p}}$.
Relation \ref{satadj1} is then solved using Newton-Raphson iterations while updating $q_{l}$, $q_{v}$, $T$ and $\Theta_{l}$ after each step. The system to be solved can be written:
\begin{equation}
q_{l}^{m+1} = q_{l}^{m} + \frac{q_{t} - q_{l}^{m} - q_{s}\left(P,T^{m}\right)}{1 + \frac{\partial q_{s}\left(P,T^{m}\right)}{\partial q_{l}^{m}}},
\end{equation}
where $m$ denotes the current Newton step. Iterations are stopped when $\left|q_{l}^{m+1} - q_{l}^{m}\right| < \epsilon$, and we set $q_{l}^{*} = q_{l}^{m+1}$.

When the saturation adjustment scheme is applied, only the mixing ratio of cloud water is updated. It is thus assumed that all the water vapor in excess condenses over cloud
droplets only. Rain mixing ratio is thus taken into account to evaluate the total water mixing ratio necessary for the resolution of the above system, but it is not modified
after the procedure is completed. 

When considering mixed-phase clouds, the method can still be applied with minor biases in liquid dominated clouds but large errors can appear when the ice/liquid equilibrium
lies far from saturation over liquid, that is when the ice proportion within the cloud layer becomes large. In such conditions, an explicit treatment of condensation/deposition 
sources is required.

Saturation adjustment schemes for mixed-phase environments have been proposed in the litterature, but none of these solutions yield reasonable results. They usually fail to predict
the right repartition between liquid and ice phase. 
As a result, when mixed-phase clouds are to be simulated using MIMICA, the direct integration of all condensation/sublimation sources
is employed.

Note that the saturation adjustment scheme is always used prior to the first time step to initiate the cloud layer. The initial cloud cover/liquid water field is indeed usually 
diagnosed from the input potential temperature and total water mixing ratio soundings using the saturation adjustment method.

\subsubsection{Direct integration of condensation/evaporation}

When condensation/evaporation sources (respectively deposition/sublimation) are explicitly integrated, the implicit dependence of the diffusional sources on supersaturation
over water and ice may be the source of numerical issues. The diffusional growth terms are simplified for the present demonstration:
\begin{equation}
C_{k} = \frac{s}{\tau_{lk}}  \mbox{\hspace{2cm} (} D_{k} = \frac{si}{\tau_{ik}} \mbox{)},
\label{evap}
\end{equation}
with $s = q_{v}-q_{s}$ ($si = q_{v} - q_{si}$) being the supersaturation over liquid water (ice), and $\tau_{lk}$  ($\tau_{ik}$) are characteristic time scales of 
condensation/evaporation (deposition/sublimation) processes that can be deduced from relations given in \ref{phasetrans}. A detailed resolution of supersaturation evolution, 
which is necessary for a proper representation of condensation/evaporation processes, requires time-steps of the order of milliseconds. As this is not conceivable in LES of 
atmospheric flows, specific alternatives must be found to preserve the system stability as well as the accuracy of the solution. 

In the present LES code, the strong implicit relation between supersaturation and condensation/evaporation is modeled using a pseudo-analytic solution of the supersaturation
equation over a time step, as proposed for example by Morrison and Grabowski \cite{MG2007}. For the purpose of the demonstration, we simplify the entire system to account only
for diffusional growth (mixing and advection were neglected):
\begin{eqnarray}
\frac{\partial q_{l}}{\partial t} & = & \frac{s}{\tau_{l}} \label{condevap1}\\
\frac{\partial q_{i}}{\partial t} & = & \frac{si}{\tau_{i}} \\
\frac{\partial q_{v}}{\partial t} & = & - \frac{s}{\tau_{l}} - \frac{si}{\tau_{i}}
\label{condevap2}
\end{eqnarray}
The system can be completed by the following relations:
\begin{equation}
\frac{\partial q_{s}}{\partial t} = \frac{\partial q_{s}}{\partial T}\frac{\partial T}{\partial t} = -A.
\end{equation}
\begin{equation}
\frac{\partial q_{si}}{\partial t} = \frac{\partial q_{si}}{\partial T}\frac{\partial T}{\partial t} = -B.
\end{equation}
where we define $\frac{\partial T}{\partial t}$, a temperature tendency term containing radiation, turbulent mixing as well as advection through updrafts/downdrafts.
Combining the previous relations, it is possible to derive balance equations for $s$ and $si$. The equation for $s$ and $si$ can be written as follows:
\begin{equation}
\frac{\partial s}{\partial t} = - \frac{s}{\tau} + \frac{1}{\tau_{i}}\left(s - s_{i}\right) + A
\label{equa_s}
\end{equation}
\begin{equation}
\frac{\partial si}{\partial t} = - \frac{si}{\tau} - \frac{1}{\tau_{c}}\left(s - s_{i}\right) + B ,
\label{equa_si}
\end{equation}
with the characteristic phase relaxation time:
\begin{equation}
\tau = \left(\frac{1}{\tau_{l}} + \frac{1}{\tau_{i}}\right)^{-1},
\end{equation}

Equation \ref{equa_s} (respectively \ref{equa_si}) can be integrated under the assumption that $\tau_{c}$, $\tau_{i}$ and $A$ remain constant during time $t$:
\begin{equation}
s\left(t\right) = \tau�\left[\frac{s_{i}}{\tau_{i}} + \frac{s}{\tau_{c}} - \tau\left(\frac{B}{\tau_{i}} + \frac{A}{\tau_{c}}\right)\right]e^{-\frac{t}{\tau}}
+ \tau^{2}\left(\frac{B}{\tau_{i}} + \frac{A}{\tau_{c}}\right) + \frac{\tau}{\tau_{i}}\left[s - s_{i} + \left(A - B\right)t\right].
\label{snp1}
\end{equation}
A very similar relation can be found for $si$. Compared to the work of Morrison et al. \cite{Mal2005}, the dependence of $s-s_{i}$ on time has been considered. This yields
a more complicated but more accurate solution of the supersaturation equation.

Condensation/evaporation (sublimation/deposition) sources are now evaluated using a time averaged supersaturation integrated over a timestep. Integrating relation 
\ref{snp1} over $\Delta t$ gives
\begin{eqnarray}
\overline{s} & = & \frac{1}{\Delta t}\int_{t}^{t+\Delta t} s\left(t^{*}\right) dt^{*} \\
 & = & \frac{\tau^{2}}{\Delta t}�\left[\frac{s_{i}}{\tau_{i}} + \frac{s}{\tau_{c}} - \tau\left(\frac{B}{\tau_{i}} + \frac{A}{\tau_{c}}\right)\right]\left(1 - e^{-\frac{\Delta t}{\tau}}\right)
+ \tau^{2}\left(\frac{B}{\tau_{i}} + \frac{A}{\tau_{c}}\right) + \frac{\tau}{\tau_{i}}\left[s - s_{i} + \frac{\Delta t}{2}\left(A - B\right)\right].
\label{mean_s}
\end{eqnarray}
The mean supersaturation obtained above is then used to calculate the condensation/evaporation sources for liquid drops whereas a mean ice supersaturation (taking a
similar form) is used to evaluate the deposition/sublimation sources of all ice particles.

Although quite complicated, the method provides a very stable and accurate way of integrating condensation/evaporation sources when a detailed evolution
of supersaturation is required. The issue related to supersaturation integration in high resolution models appears to be a non trivial one and finding efficient methods to
overcome the issue (not mentioning the saturation adjustment scheme) is not a simple task. 

The model is selected by setting lgrowth to 1 in {\it cm.nml}. If lgrowth is turned to 0, supersaturation for diffusional growth is evluated explicitly at time $t$.

\subsubsection{Droplet activation}

When the chemistry/aerosol modules are deactivated, an empirical parameterization of the number of activated CCN is chosen, based on a power law of the supersaturation:
\begin{equation}
N_{CCN} = N_{CCN,0} S^{\kappa},
\end{equation}
with $N_{CCN}$ the number of activated CCN and $N_{CCN,0}$ and $\kappa$ two constants. $N_{CCN,0}$ is usually defined as the background number of aerosols that can be
activated, and $\kappa$ varies between $0.3$ to $0.7$ depending on the case. The supersaturation is here defined as:
\begin{equation}
S = \frac{q_{v} - q_{s}}{q_{s}}\times 100,
\end{equation}
where $q_{v}$ is the vapor mixing ratio and $q_{s}$ the saturation mixing ratio. Supersaturation is expressed in \% and is explicitly limited to a value of $1\%$ in
the code (beware if larger values are expected). Note that $S$, the relative supersaturation, must not be confused with $s$, the absolute supersaturation (used in the
diffusional growth theory above).

The number of droplets formed is explicitly assumed to be equal to the number of activated CCN: $\left.N_{c}\right|_{nuc} = N_{CCN}$. All the droplets thus formed have the same
mass which is equal to 4.19e15 kg (the mass of a droplet having the minimum diameter possible). 

\subsubsection{Ice nucleation}
\label{icenuc}

At the moment, only a simple ice nucleation scheme is present in MIMICA, as used in the ISDAC intercomparison for instance. Ice nucleation is defined as a simple relaxation 
of the ice number concentration towards a fixed background IN concentration:
\begin{equation}
\left.\frac{\partial N_{i}}{\partial t}\right|_{nuc} = \frac{N_{i} - IN_{0}}{\Delta t}.
\end{equation}
To limit the nucleation events, this relation is only applied where liquid water is also present and where supersaturation over ice is sufficient, for example where $Si > 5 \%$.
All the newly nucleated ice crystals are assumed to have the same mass, defined as the minimum ice crystal mass possible. It results that:
\begin{equation}
\left.\frac{\partial Q_{i}}{\partial t}\right|_{nuc} = m_{min} \left.\frac{\partial N_{i}}{\partial t}\right|_{nuc}.
\end{equation}
At the moment, $m_{min}$ is hard coded but can be changed easily in {\it micro\_ice.f90}.

On a long term basis, a detailed ice nucleation scheme should be implemented in MIMICA. The model will be based on nucleation theory and follow state-of-the-arts parameterization
that can be commonly found in the litterature. In such works, all the different nucleation paths are usually included, namely nucleation via deposition, immersion, contact and 
condensation/freezing. 

\subsection{Boundary conditions}

\subsubsection{Lateral boundaries}
By default, lateral boundary conditions are periodic which requires the definition of one or two (depending on the stencil used by the spatial discretization) ghost cells
on both sides of the domain and in each direction. The values in the first or first two cells inside the domain at the inlet (respectively outlet) are copied into the
one or two ghost cells present at the outlet (respectively inlet). This is done either by direct recopy (sequential job) or MPI communication (parallel job). 
The existence of ghost cells in both horizontal directions implies that all the 2D/3D arrays used in the code must be allocated accordingly. One has thus to be very
careful when allocating new arrays or when exchanging data between different domains.

\subsubsection{Top boundary}
The standard treatment of the top boundary does not require any particular operation or model. In the last vertical layer of the domain, we will indeed consider that all
the turbulent fluxes vanish (which is a reasonable assumption) so that we are only interested in the local sources and advection terms. Local sources do not
present any difficulty as this is the case also for horizontal advection. Vertical advection by the $w$ velocity can however lead to some problems, especially if we don't
know if $w$ is locally positive (directed upward) or negative (downward). In the first case, the flow goes out of the domain so that if the system of equation is well
posed, it can be shown that the solution in the top layer is only determined by the initial condition (we first assume that there is no gravity wave). In other words, the
flow goes out and no additional information is required to treat the boundary. On the opposite, if $w$ is negative, it then appears that the flow comes inside the domain
through the top boundary so that some values must be imposed at the "inlet" (otherwise the system would be ill-posed because information from outside the domain is 
required). In that case, the last vertical layer of the domain for k=nz is actually a virtual layer (as is the first layer at the surface) in which all the quantities are
held constant and equal to their original values. Using these quantities, it is possible to evaluate gradients in the last physical layer in order to calculate the
vertical advection. 

When considering the top boundary, things are actually not as simple as that. A thorough analysis of the system of equations solved by the model can show that gravity
waves can freely propagate inside the domain and will interact with inflow/outflow boundaries if we don't pay any particular attention. Indeed, classical boundaries 
act numerically as rigid layers on which gravity waves are able to reflect and thus propagate back into the domain. A specific treatment must then be applied in order
to avoid such kind of spurious wave reflexion. 

The simplest way to get rid of these gravity waves is to apply a sponge layer in the last 3-5 layers (controled by {\it zdamp} in {\it cm.nml}) of the domain in order to 
dampen any incoming wave. The damping function used in the model allows a fast relaxation of the local solution towards the initial reference state and is defined as:
\begin{equation}
\Psi_{k} = \Psi_{k} - \left(1 - \frac{z_{top} - z_{k}}{H}\right)\frac{\Psi_{k} - \Psi_{k}^{0}}{\tau_{sl}}.
\end{equation}
The time scale $\tau_{sl}$ can be changed in {\it cm.nml} and should be at least of the order of 100 s. $H=z_{top}-z_{damp}$ is the height of the sponge layer whereas 
$z_{top}$ is the altitude of the last scalar point in the domain. The factor depending on the altitutde allows a linear increase of the strength of the relaxation up 
to the top of the domain. The damping is applied to all the transported scalars and winds.

A more sophisticated approach consists in identifying exactly the gravity wave at the top layer and simply canceling the incoming component. In that case, it is common to
define a simplified system of equation (non-hydrostatic, non-elastic, which remains reasonable at the top boundary) and to perform an analysis in the spectral space where
waves are easily identified. A simple condition relating the vertical velocity and the pressure in the spectral space can finally be obtained that will allow upward waves
to go out properly while damping downward gravity waves. Details of the derivation of so-called radiation boundary conditions can be found for example in Bougeault 
\cite{Bou1983}. The condition on the pressure can be expressed as:
\begin{equation}
\hat{\pi} = \frac{N}{\sqrt{k^{2} + l^{2}}}\hat{w}.
\label{radiation}
\end{equation}
The previous relation was written for the perturbation Exner pressure $\pi$ instead of the absolute pressure to remain consistent with the model formulation. $k$ and $l$
are the horizontal wave numbers in the $k$ and $l$ directions. The condition expressed above can be seen also as a dynamics adaptation of the top layer's height (in
pressure coordinate) in response to incoming gravity waves (the top layer moves up and down with incoming waves).

The implementation of such a radiation boundary condition is simplified in MIMICA as a spectral solver is used to obtain a diagnostic pressure through the Poisson
equation. Let's recall here that the diagnostic pressure is obtained by solving a tridiagonal system in the vertical direction while working in the spectral space in the
horizontal plane. The solution of the tridiagonal system is then $\hat{p}$ so that condition \ref{radiation} will appear only as a fixed boundary condition for
the tridiagonal system at the top layer. The only remaining requirement is thus to obtain the transformed $\hat{w}$ in the horizontal spectral space, which can be done
easily by calling the available FFT package (done only on the master processor, as for $p$). The radiation boundary condition can be selected through the keyword 
RADIA\_BC in {\it start}.

\subsubsection{Surface modeling}
In LES simulations of the atmospheric boundary layer, the treatment of the bottom boundary is crucial. It must indeed provide a source of moisture and heat through
evaporation processes and act as a rough surface to slow down the winds. Modeling of surface exchanges are mainly treated through the evaluation of the surface
turbulent fluxes.

Symmetry is assumed in the first vertical point of the domain for velocities (slip condition):
\begin{equation}
{\mathbf u}\left(x,y,z=z_{1}\right) = {\mathbf u}\left(x,y,z=-z_{1}\right).
\end{equation}
Note that the point at $z=-z_{1}$ is actually stored in the first layer of the grid corresponding to $k=1$. After each momentum or scalar integration, the values in the
second vertical mesh points are therefore copied into the first vertical cell. This means that the physical domain actually starts from the second vertical grid layer.
Regarding scalars, symmetry is also imposed in the first vertical point:
\begin{equation}
\Psi\left(x,y,z=z_{1}\right) = \Psi\left(x,y,z=-z_{1}\right). 
\end{equation}
These two conditions are important to evaluate vertical gradients in the first layers of the domain.

The modeling of turbulent surface fluxes is controled through the parameter {\it isurf} present in {\it cm.nml}. {\it isurf} is defined as a double digit integer for which
the first digit controls the method to obtain the surface friction quantities and the second digit, the surface model (fixed SST or fixed surface fluxes). 
{\it isurf} can take the following values 10, 11, 12, 20, 21 or 22. 

Momentum and virtual potential temperature fluxes at the ground are evaluated using a friction velocity such as:
\begin{equation}
u_{*}u_{*} = \overline{u^{\prime}w^{\prime}} = \overline{v^{\prime}w^{\prime}}.
\label{sfmom}
\end{equation}
\begin{equation}
u_{*}\theta_{v*} = \overline{\theta_{v}^{\prime}w^{\prime}}.
\label{sftem}
\end{equation}
The virtual potential temperature fluxes (idrectly related to buoyancy fluxes) are used to retrieve potential temperature and moisture surface fluxes. 
The two models available through {\it isurf} are given hereafter: $u_{*}$ can either be fixed to a constant value when {\it isurf} equals 20, 21 or 22 (default is $0.25$) 
or it can be computed dynamically using similarity theory for {\it isurf} = 10 , 11 or 12. In the latter case, we must distinguish between fixed surface fluxes (isurf=11, 12) and 
fixed scalar surface values (isurf=10). When considering fixed surface fluxes, the sensible and latent heat fluxes must be provided explicitly by the user through parameters 
shf and lhf respectively in {\it cm.nml}. In a similar fashion, when fixed surface quantities are considered, the parameters sst and ssm must be given the appropriate values
(when ssm = 0, skin surface moisture is retrieved assuming 100\% saturation over liquid or ice at the skin surface temperature).

We write the turbulent flux $\overline{u^{\prime}w^{\prime}}$ following:
\begin{equation}
\overline{u^{\prime}w^{\prime}} = -K_{m} \frac{\partial u}{\partial z},
\end{equation}
with $K_{m}$ approximated by:
\begin{equation}
K_{m} = \kappa z u_{*},
\end{equation}
where $\kappa$ is the von Karman constant ($\approx$ $0.35$) and $z$ is the altitude of the first velocity grid point. Combining these relations and integrating over $z$
yields:
\begin{equation}
ln\left(\frac{z}{z_{0}}\right) = \frac{\kappa \left| u \right|}{u_{*}},
\label{neutral}
\end{equation}
from which $u_{*}$ can be deduced. Here, $z_{0}$ is a surface roughness height depending on the ground nature (set to $1.e-4$ m by default).
Relation \ref{neutral} remains valid as long as the surface layer remains neutral. It doen't account for any TKE production through buoyancy.

For stable and unstable BL, Monin-Obukhov similarity theory can be used to derive a relation similar to \ref{neutral} (after integration along z). In these cases we
must define the surface buoyancy flux $\overline{\theta_{v}^{\prime}w^{\prime}}$ from the input surface parameters. The buoyancy flux will thus take different forms
depending on which surface model is used: fixed SHF and LHF or fixed SST. In the case were surface heat fluxes are prescribed, we written:
\begin{equation}
b = \frac{g}{\theta_{00}}\overline{\theta_{v}^{\prime}w^{\prime}} = \frac{g}{\Theta_{00}}\left(\frac{SHF}{cp} + 0.608 \theta_{00}\frac{LHF}{L_{v}}\right).
\end{equation}
We then define the Monin-Obukhov length scale by:
\begin{equation}
L = - \frac{u_{*}^{3}}{\kappa b}.
\end{equation}
Using these relations, the friction velocity can be deduced from a relation of the form:
\begin{equation}
ln\left(\frac{z}{z_{0}}\right) - \Psi_{m}\left(\zeta\right) = \frac{\kappa \left| u \right|}{u_{*}}.
\label{noneutral}
\end{equation}
This equations is similar to \ref{neutral} derived for neutral surface layers, but the term $\Psi_{m}$ appears now to account for surface buoyancy fluxes, source of stability.
The ratio $\zeta = z/L$ can be viewed as a measure of ABL stability (stable for $\zeta>0$ and unstable for $\zeta<0$). $\Psi_{m}$ is commonly deduced 
from observations and depends on the sign of $\zeta$. We use here the forms proposed by Garratt \cite{Gar1992}:
\begin{eqnarray}
\mbox{for } \zeta > 0 & \mbox{      } & \Psi_{m}\left(\zeta\right) = -5\zeta \\
\mbox{for } \zeta < 0 & \mbox{      } & \Psi_{m}\left(\zeta\right) = 2 ln\left(\frac{1+x}{2}\right) + ln\left(\frac{1+x^{2}}{2}\right) - 2 tan^{-1}\left(x\right) +
\frac{\pi}{2} \mbox{\hspace{1cm} and \hspace{1cm}} x = \left(1 - 16\zeta\right)^{1/4}.
\end{eqnarray}
As this was the case for the neutral boundary layer, $u_{*}$ can be derived from relation \ref{noneutral}. However, $L$ depending directly on $u_{*}$, an iterative
procedure is required to obtain a converged value of $u_{*}$, with a first guess given by the neutral formula. Five subiterations are used. The friction virtual potential
temperature is finally deduced after the last subiteration from: $\theta_{v*} = - \frac{\theta_{00} b}{g u_{*}}$. This quantity is actually not used in the calculation
of the surface fluxes as they are readily provided by SHF and LHF. 

In the second case, when skin surface scalars are prescribed, the surface buoyancy flux is calculated from the gradient between the virtual 
potential temperature in the first LES grid cell above the surface and a skin surface virtual potential temperature:
\begin{equation}
b = \frac{g}{\theta_{00}}\left[\left(\theta_{1} - \Pi_{00} SST\right) + 0.608 \theta_{00}\left(q_{t1} - SSM\right)\right] = \frac{g}{\theta_{00}}\left(\theta_{v1} - \theta_{v,skin}\right).
\end{equation}
An initial friction potential temperature can readily be obtained under neutral conditions following:
\begin{equation}
ln\left(\frac{z}{z_{0}}\right) = \frac{k b \theta_{00}}{g Pr_{t} \theta_{v*}},
\end{equation}
with $Pr_{t}$ the turbulent Prandtl number defined in {\it cm.nml}. We can now define as previously the Monin-Obukhov characteristic length scale, but using the surface buoyancy flux
defined above:
\begin{equation}
L = \frac{u_{*}^{2} \theta_{00}}{\kappa g \theta_{v*}}.
\end{equation}
From these relations, the friction velocity and virtual potential temperature can be deduced from similar relations as previously:
\begin{equation}
ln\left(\frac{z}{z_{0}}\right) - \Psi_{m}\left(\zeta\right) = \frac{\kappa \left| u \right|}{u_{*}},
\end{equation}
\begin{equation}
Pr_{t}\left[ln\left(\frac{z}{z_{0}}\right) - \Psi_{h}\left(\zeta\right)\right] = \frac{\theta_{00} b}{g \theta_{v*}},
\end{equation}
with the ratio $\zeta = z/L$ defined as previously. Whereas $\Psi_{m}$ doesn't change compared to the fixed surface fluxes case, $\Psi_{h}$ is evaluated in a very similar way as
a function of $\zeta$ and depending on its sign:
\begin{eqnarray}
\mbox{for } \zeta > 0 & \mbox{      } & \Psi_{h}\left(\zeta\right) = -7\zeta \\
\mbox{for } \zeta < 0 & \mbox{      } & \Psi_{h}\left(\zeta\right) = 2 ln\left(\frac{1+y}{2}\right) \mbox{\hspace{1cm} and \hspace{1cm}} y = \left(1 - 9\zeta\right)^{1/2}.
\end{eqnarray}
Once again, the implicit dependence on $u_{*}$ and $\theta_{*}$ requires subiterations to yield an accurate solution. Now, both friction scalars are updated at the end of each substep.
Once the final value of $\theta_{v*}$ is found, the surface fluxes of both temperature and moisture can be retrieved by simply assuming the same contribution of absolute temperature and
moisture to the surface virtual potential temperature flux as for the skin surface virtual potential temperature. Therefore, SST and SSM are not used directly to compute the surface fluxes, 
but they are used in the Monin-Obukhov theory to determine the surface friction quantities that are in turn used in the surface fluxes expressions following \ref{sfmom} and \ref{sftem}.

In the MIMICA model the first layer above the ground is given by the second cell
layer in the vertical dimension, for k=2. The layer k=1 actually corresponds to a virtual layer used to store surface quantities and therefore calculate gradients in the
first layer. 

\subsection{Large scale tendencies (nudging)}

In some cases, it can be requested that the simulated wind and scalar fields must be nudged to a given reference state.
If we consider one particular quantity $\Phi$, nudging is treated by simply adding one extra term $N_{\Phi}$ in the corresponding balance equation. This term allows the
relaxation of the $\Phi$-field towards the reference state $\Phi_{0}$ with a given time scale $\tau_{n}$. We then write:
\begin{equation}
N_{\Phi}\left(x,y,z;t\right) = -\frac{\Phi\left(x,y,z;t\right) - \Phi_{0}\left(x,y,z;t\right)}{\tau_{n}}.
\end{equation}
The relaxation time scale $\tau_{n}$ is usually case and variable dependent. It can be a constant, depend on the vertical coordinate or on the simulation time. The
reference state $\Phi_{0}$ is often taken to be one-dimensional only (dependent on the height only) and equal to the initial state.

In a similar fashion, large scale scalar advection can be considered. Large scale advection is simply modeled by a constant source added to the scalar tendencies (only
for ice/liquid potential temperature and total water mixing ratio). These sources can possibly be time dependent (to be defined in {\it timevar.f90}).

\section{Numerics}
\label{numerics}

\subsection{Time integration} 
\label{time_step}

In the current version of the model, four integration methods for the dynamics part are available: the 1st order Euler forward scheme (unstable for $CFL < 0.5$, but 
useful for defbugging purposes), a 2nd order Leap-Frog method with temporal filtering, a 3rd order TVD Runge-Kutta scheme and a 4th order low-storage, low-dispersion 
and low-dissipation Runge-Kutta scheme. The three latter methods are described below.

Considering a dummy variable $\Psi$ whose values are known at times $n$ and $n-1$, its value at time $n+1$ estimated through the Leap-Frog method will be given by
(semi-discretized form):
\begin{equation}
\frac{\Psi^{n+1} - \Psi^{n-1}}{2\Delta t} = {\mathcal F^{n}}\left(\Psi\right),
\end{equation}
where ${\mathcal F^{n}}$ is the local flux of $\Psi$ evaluated at time $n$. The Leap-Frog method is second order accurate and has a stability criterion that can go up
to about $0.8$ in 3D depending on the choice of spatial discretization. Note that as Leap-Frog time integration requires the knowledge of all scalar fields at
time $n-1$, the method cannot be used to advance the first time step. Initialization of the solution at $n=1$ must hence be done by advancing the first time step thanks 
to the eplicit forward Euler method. As this is done only for the first time step, it does not significantly affect the solution's accuracy and stability.

Although quite interesting, the Leaf-Frog scheme possesses one important drawback. The fact that the integration procedure does not 
use the value of $\Psi$ at time $n$ may lead to a separation between the solutions at time $n$ and $n+1$. Two solutions may thus exist at the same time level which
can drift away from each other due to numerical errors. To avoid this, temporal smoothing of the solution following Asselin's filter is commonly used. This ensures 
that the solution remains consistent but it also dramatically decreases the order of accuracy by adding what can be interpreted as temporal diffusion. Asselin's filter
can be written as:
\begin{equation}
\Phi^{n} = \Phi^{n} + \nu\left(\Phi^{n-1} - 2\Phi^{n} + \Phi^{n+1}\right)
\label{asselin}
\end{equation}
The coefficient $\nu$ is fixed at $0.3$ in MIMICA. Recently, an improvement of the method was proposed by Williams \cite{Wil2009} to reach effectively 2nd
order accuracy. The improved RAW filter is very simple: instead of relaxing only the solution at time $n$, the solution at time $n+1$ is also altered in such a way that the sum 
of the three successive available solutions $n-1$, $n$ and $n+1$ is conserved during the damping process. If we note $\alpha$ the term added to $\Phi^{n}$ on the right hand side 
of relation \ref{asselin}, then the improved RAW filter reads:
\begin{eqnarray}
\Phi^{n} & = & \Phi^{n} + k \alpha \\
\Phi^{n+1} & = & \Phi^{n+1} + \left(k - 1\right)\alpha
\end{eqnarray}
Following \cite{Wil2009}, the optimal value of $k$ is equal to $0.5$. In the model, a value of $0.55$ is fixed. Note that for $k=1$, the classical Asselin method is retrieved. 
The filter is controled by the nsmooth parameter in {\it cm.nml}: for $nsmooth=1$ the classical Asselin filter is used, for $nsmooth = 2$, the RAW filter is used. 

The 3rd order TVD Runge-Kutta method implemented in MIMICA can be found in, for instance, Gottlieb and Shu \cite{GS1998}. A TVD method (for "Total Variation Diminishing",
see scalar advection section below) has the property that the total variation of any transported quantity (that is the sum over all cells of the absolute difference between two successive
cell values) does not increase from one time step to another. This condition ensures that the scheme is necessarily monotonicity preserving so that $\Phi^{n}_{i+1} \leq \Phi^{n}_{i}$.
This guarantees that no spurious numerical oscillations will develop and the solution will remain free of purely numerical instabilities. Of course the numerical methods include both
the time integration scheme and the advection schemes and both must must be TVD to guarantee the TVD condition for the overall integration step. 

Following \cite{GS1998}, the 3rd order TVD Runge-Kutta can be expressed under the condensed form:
\begin{equation}
\Psi^{m} = \alpha_{m1} \Psi^{n} + \alpha_{m2} \Psi^{m-1} + \beta_{m} \Delta t {\mathcal F^{m-1}},
\end{equation}
with $m$ varying from 1 to 3. The initial conditions are given by: $\Psi^{0} = \Psi^{n}$ and $\mathcal{F}^{0} = \mathcal{F}\left(\Psi^{n}\right)$. The optimal values for coefficients 
$\alpha_{m1}$, $\alpha_{m2}$ and $\beta_{m}$ are given in table \ref{RK3}. At the end of the sub-steps, we obtain: $\Psi^{n+1} = \Psi^{3}$.

\begin{table}[!h]
\begin{center}
\begin{tabular}{c|ccc}
i & $\alpha_{i1}$ & $\alpha_{i2}$ & $\beta_{i}$ \\
\hline
1 & 1 & 0 & 1 \\
2 & 3/4 & 1/4 & 1/4 \\
3 & 1/3 & 2/3 & 2/3 \\
\end{tabular}
\end{center}
\caption{{\it Coefficients of the TVD Runge-Kutta 3 scheme.}}
\label{RK3}
\end{table}

The so-called Runge-Kutta 44-2S method, as proposed by Ketcheson \cite{Ket2010}, belongs to the familly of low-storage Runge-Kutta schemes which improve the computer memory 
usage and efficiency compared to classical Runge-Kutta formulations, while conserving the same stability and accuracy properties. It is refered as "44" because it is 4th order accurate and 
it requires only 4 sub-steps to reach this condition. The scheme can be written in a condensed form as follows:
\begin{eqnarray}
w^{m} &=& w^{m-1} + \delta_{i} \Psi^{m-1} \\
\Psi^{m} &=& \gamma_{m1} \Psi^{m-1} + \gamma_{m2} w^{m} + \beta_{m} \Delta t {\mathcal F^{m-1}},
\end{eqnarray}
where $m$ is the indice denoting Runge-Kutta sub-steps (here $m=1,...,4$). The algorithm is initialized with $\Psi^{0} = \Psi^{n}$, $\mathcal{F}^{0} = \mathcal{F}\left(\Psi^{n}\right)$ and 
$w^{0} = 0$. At the end of the sub-steps, we obtain: $\Psi^{n+1} = \Psi^{4}$. The values of the four Runge-Kutta coefficients are given in the table below.

\begin{table}[!h]
\begin{center}
\begin{tabular}{c|cccc}
i \hspace{1cm} & $\delta_{i}$ & $\gamma_{i1}$ & $\gamma_{i2}$ & $\beta_{i}$ \\
\hline
1 & 1.0000000 & 0.0000000 & 1.0000000 & 1.1937439 \\
2 & 0.2176833 & 0.1210985 & 0.7217817 & 0.0992799 \\
3 & 1.0658413 & -3.8438337 & 2.1212093 & 1.1316780 \\
4 & 0.0000000 & 0.5463789 & 0.1986530 & 0.3106657 \\
\end{tabular}
\end{center}
\caption{{\it Coefficients of the Runge-Kutta 4S scheme.}}
\label{RK}
\end{table}

Like every Runge-Kutta methods, the two present schemes require the evaluation of all scalar sources and advection fluxes three/four times within each time step. Compared to the
Leap-Frog method, the CPU cost of the time integration is thus multiplied by a factor 3/4. However, whereas Leap-Frog, coupled with Asselin's filtering method, has an
order of accuracy around 1, the present Runge-Kutta methods are respectively 3rd and 4th order accurate and do not need any filtering.
Note that due to the necessity of performing successive subiterations within one Runge-Kutta step, time advancement of winds and scalars cannot be decoupled and  
must be performed simultaneously during each RK substep. Consequently, whereas winds and scalars are integrated in two separate routines for the Leap-Frog method, 
they are gathered in a single routine when either Runge-Kutta scheme is selected. 

\subsection{Numerical grid}

The mesh used by the LES model is defined in a cartesian coordinate only, in 2 or 3 dimensions. The cell size is fixed constant in the horizontal plane but some refinement
and stretching can be applied in the vertical direction. This operation is performed in {\it refine.f90} where user-defined functions can be entered to characterize the
non-constant grid spacing. Some basic functions already exist and can be used at will. 

The domain length and number of cells in all 3 directions 
can be set by the user in the {\it start} file. Note that extra cells were initially defined for each velocity component in their respective directions. Although they are 
no longer used in the model, they still exist and need to be considered. 

Cartesian grids are very convenient to discretize fluid equations using finite differences. In the present code, a staggered grid system is employed, the
so-called C-Arakawa grid, defined as follows:
\begin{itemize}
\item[$\bullet$] pressure, $p_{ijk} = p\left(x,y,z\right)$, and all the other scalars, are cell-centered
\item[$\bullet$] the three velocity components are shifted by half a cell size downtream of $p$ in their respective direction: 
$u_{ijk} = u\left(x+\frac{\Delta x}{2},y,z\right)$, $v_{ijk} = v\left(x,y+\frac{\Delta y}{2},z\right)$, $w_{ijk} = w\left(x,y,z+\frac{\Delta z}{2}\right)$
\end{itemize}
Using such a staggered grid system can largely increase the accuracy of the advection scheme without requiring the use of high order schemes \cite{Dur2010}. In addition,
the local velocity (respectively pressure) gradients will be defined exactly at the point where pressure (respectively velocity) is defined, using a simple first order
approximation without interpolation. This improves the code's readability by making the evaluation of such gradients straightforward.

In practice, due to grid staggering, two different sets of vertical grid spacings have to be defined when vertical refinement is used: one corresponding to the spacing
between two vertical velocity points, the other defined as the spacing between two scalar points. These two quantities will be refered to as $\Delta z_{w}$ and $\Delta
z_{p}$ respectively.

\subsection{Momentum advection}

In LES models, and more generally in every small or large scale atmospheric models, the advection scheme is always considered as a key element that needs to be treated
carefully. As explained in the next section, conservation of scalars and momentum during the advection step is enforced by the inversion of a Poisson equation for pressure 
that ensures that the flow remains divergence free everywhere. We start by describing momentum advection and then scalar advection which requires additional constraints.

The momentum advection scheme must provide sufficient accuracy and stability when coupled with the integration scheme. With no additional constraint, this leads to a very 
limited choice, mainly based on the easiness of implementation. Based on this, we chose to discretize momentum advection terms following central finite differences.
These methods usually provide good accuracy given that they generate no numerical diffusion, and are in addition easy to implement.
For clarity sake, the advective fluxes are recast into a conservative form:
\begin{equation}
u\frac{\partial u}{\partial y} \equiv \frac{F_{i,j+1/2} - F_{i,j-1/2}}{\Delta y},
\end{equation}
where $F_{i,j+1/2}$ and $F_{i,j-1/2}$ are the fluxes interpolated respectively at the center of the right and left cell faces on the C-Arakawa grid (in the $y$ direction). 
For simplicity, we have here considered only a 2D system where the subscript $i$ is used in the $x$ direction and $j$ in the $y$ direction. 
This formulation is by definition energy and mass conserving and will therefore provide a better stability than any other formulation. The issue now consists in evaluating
the interpolated fluxes. If we consider polynomial interpolations at the cell centers, the interpolated fluxes can be expressed as:
\begin{equation}
F_{i,j+1/2} = \sum_{l=0}^{n-1}C_{a,l}f\left(u_{i,j-a+l}\right),
\end{equation}
where $n$ represents the order of the interpolation, $a$ the first point considered for the interpolation, $f$ the local flux and $C_{a,l}$ the coefficients of the 
polynom. Selecting $a=1$ and $n=4$  yields a 4th order central approximation for the interpolated flux given by:
\begin{equation}
F_{i,j+1/2} = \frac{1}{2}\left(v_{i+1,j} + v_{i,j}\right)\left[-\frac{1}{12} u_{i,j-1} + \frac{7}{12} u_{i,j} + \frac{7}{12} u_{i,j+1} - \frac{1}{12} u_{i,j+2}\right].
\end{equation}
Considering the grid is staggered, an interpolation of the advective velocity $v$ at the cell face is required. For the 2nd order central scheme, the above formulation
becomes:
\begin{equation}
F_{i,j+1/2} = \frac{1}{2}\left(v_{i+1,j} + v_{i,j}\right)\left[\frac{1}{2} u_{i,j} + \frac{1}{2} u_{i,j+1}\right],
\end{equation}
where we set $a=0$ and $n=2$. In practice, it is possible to switch from 2nd order to 4th order by setting the ADV4 parameter in {\it start} to TRUE. 

The biggest advantage of central schemes is also their biggest drawback. Their lack of numerical dissipation may indeed lead to the amplification of small
dispersive errors that are free to propagate within the field and cause numerical instabilities. It is therefore often necessary to filter these short wavelengths in order
to damp the possibly growing instabilities. Following Durran \cite{Dur2010}, a 2nd order or 4th order filter operator is applied to horizontal advection of momentum
to add artificial diffusion. We write:
\begin{equation}
v\frac{\partial u}{\partial y} \equiv \mathcal{F}^{n}_{i,j} + \gamma_{n} \widehat{u_{i,j}}^{nj},
\label{filter}
\end{equation}
where $\mathcal{F}^{n}_{i,j}$ represents the centered nth order ($n$ equals 2 or 4 here) semi-discrete advection operator, and $\widehat{.}^{nj}$ denotes the nth order 
spatial filter applied along $j$ defined by:
\begin{equation}
\widehat{u_{i,j}}^{4j} = u_{i,j+2} - 4u_{i,j+1} + 6u_{i,j} - 4u_{i,j-1} + u_{i,j-2}.
\end{equation} 
and:
\begin{equation}
\widehat{u_{i,j}}^{2j} = -u_{i,j+1} + 2u_{i,j} - u_{i,j-1}.
\end{equation} 
On the staggered grid system, the filtered terms are expressed at the same locations as the velocity vector components. In \ref{filter}, $\gamma_{n}$ represents the strength
of the filter. If we fix $\gamma_{4} = \left|u_{i,j}\right| / 12\Delta x_{j}$ (resp. $\gamma_{2} = \left|u_{i,j}\right| / 2\Delta x_{j}$) along with the use of 4th order
(resp. 2nd order) central differences, the resulting advection scheme (including artificial diffusion) is exactly equivalent to the 3rd order (resp. 1st order) upwind 
scheme. In practice, we multiply $\gamma_{4}$ or $\gamma_{2}$ as described previously by a constant $C_{diff}$, with $0 \leq C_{diff} \leq 1$ (in {\it cm.nml}) 
controling the strength of artificial diffusion.

\subsection{Scalar advection}

Regarding scalar advection, additional constraints must be considered in order to make sure that the advected quantities always remain within prescribed bounds and do not
develop spurious numerical instabilities or reach unphysical values. For instance, it is obvious that mass mixing ratios (for vapor or hydrometeors) must always remain 
positive thus requiring a so-called positive-definite scheme. It must also be realized that low dissipation central difference schemes, as presented above for momentum 
advection, may develop numerical oscillations (wiggles) in regions where large gradients are encountered. This is particularly the case in the vicinity of an inversion 
layer. Consequently, specific advective schemes must be employed for scalars to avoid the development of instabilities. 
In the present model, a wide variety of schemes was implemented giving the user a large choice of methods, each having its particularities. All the schemes are at 
most 3rd order accurate which implies that they all generate little numerical dissipation. 

As for momentum advection, we use a conservative flux formulation for scalar advection:
\begin{equation}
u_{i}\frac{\partial \Phi}{\partial x_{i}} \equiv \frac{F_{i+1/2} - F_{i-1/2}}{\Delta x_{i}}.
\end{equation}
In the following are introduced the five different options available for scalar advection.
The choice of the method is controled by the keywords ADV\_LW, ADV\_TVD, ADV\_WENO and ADV\_MUSCL in {\it start}. When either of these switches is set to TRUE,the
corresponding scheme is selected. On the opposite, if they are all set to FALSE (default option), the central finite difference scheme with raw limitations is
selected. When not using the ADV4 option for momentum advection (2nd order accuracy), it is recommended to use the 2nd order Lax-Wendroff method described hereafter. 

\subsubsection{Upwind finite differences}
A first method implemented in the code relies on 3rd order upwind finite differences (it reduces to the 1st order upwind method when ADV4 is turned off in {\it start})
for which the flux balance inside each grid cell is limited to make sure that 1) no local extremum is created and 2) the transported scalar does not drop below a minimum 
value (typically $0$). In order to account for the changing velocity sign inside the domain, we define two 3rd order scalar interpolations at the faces
using two different stencils. The 3rd order interpolations are written:
\begin{equation}
\Phi_{i+1/2}^{+} = \frac{1}{6}\left(2\Phi_{i+1} + 5\Phi_{i} - \Phi_{i-1}\right),
\end{equation}
when the wind velocity along $i$ is positive, and:
\begin{equation}
\Phi_{i+1/2}^{-} = \frac{1}{6}\left(-\Phi_{i+2} + 5\Phi_{i+1} + 2\Phi_{i}\right),
\end{equation}
when the velocity is negative. The upwind fluxes are finally expressed following:
\begin{equation}
F_{i+1/2} = \frac{1}{2}\left[\left(\Phi_{i+1/2}^{+} + \Phi_{i+1/2}^{-}\right)u_{i+1/2} - \left(\Phi_{i+1/2}^{+} - \Phi_{i+1/2}^{-}\right)\left|u_{i+1/2}\right|\right].
\end{equation}
This enables to select the proper scalar interpolation at the cell face depending on the velocity sign. This 3rd order scheme is used as such in the horizontal plane.
In the vertical direction, the presence of an inversion layer sometimes requires more numerical dissipation than what is imposed by a 3rd order approximation. It is thus
possible to switch to a 1st order upwind scheme for vertical advection only, which guarantees a high stability (the selection of the vertical advection method is hard
coded at the moment and the default is the 1st order method). The finite difference formulation described here is very easy to implement and it doesn't contain any
difficulty. However, when strong discontinuities are treated and the 1st order approximation is required in the vertical direction, the scheme generates excessive
numerical dissipation which can deteriorate the solution's accuracy. In order to avoid this dramatic accuracy loss, more sophisticated methods have to be employed.

\subsubsection{2nd order flux limited Lax-Wendroff method}

Following the flux limiter approach, we seek to rewrite fluxes under the following form:
\begin{equation}
F_{i+1/2} = F^{LO}_{i+1/2} + \frac{1}{2}\phi_{i+1/2}\left(F^{HO}_{i+1/2} - F^{LO}_{i+1/2}\right),
\label{flux1}
\end{equation}
with superscripts $LO$ and $HO$ denoting low-order and high-order fluxes respectively, and $\phi$ is the flux limiter function. Therefore, the fluxes are expressed as a
combination of a high-order formulation and a monotone low order scheme (usually the upstream method) where the flux limiter $\phi$ acts as a switch selecting the monotone
scheme in the region of high gradients and the high order approximation where the solution is smooth. 

In the Lax-Wendroff method, following Durran, the upstream scheme is used as the monotone approximation and the 2nd order Lax-Wendroff method provides the basis for the
high order part. The upstream method reads:
\begin{equation}
F_{i+1/2}^{up} = \frac{u_{i+1/2}}{2}\left(\Phi_{i+1} + \Phi_{i}\right) - \frac{\left|u_{i+1/2}\right|}{2}\left(\Phi_{i+1} - \Phi_{i}\right),
\end{equation}
and the Lax-Wendroff flux can be written:
\begin{equation}
F_{i+1/2}^{lw} = \frac{u_{i+1/2}}{2}\left(\Phi_{i+1} + \Phi_{i}\right) - \frac{u_{i+1/2}u_{i+1/2}\Delta t}{2\Delta x}\left(\Phi_{i+1} - \Phi_{i}\right).
\end{equation}
Replacing the two formulae into \ref{flux1} yields:
\begin{equation}
F_{i+1/2} = \frac{u_{i+1/2}}{2}\left(\Phi_{i+1} + \Phi_{i}\right) - \frac{1}{2}\left[\left(1-\phi_{i+1/2}\right)\left|u_{i+1/2}\right| + 
\phi_{i+1/2}\frac{u_{i+1/2}u_{i+1/2}\Delta t}{\Delta x}\right]\left(\Phi_{i+1} - \Phi_{i}\right).
\label{fllw}
\end{equation}
The flux limited Lax-Wendroff method is at most 2nd order accurate in regions of smooth gradients.

Starting from the original definition of the flux limiter, $\phi$ must be defined as a function of the ratio $r$ between two successive gradients:
\begin{equation}
r_{i+1/2} = \frac{\Phi_{i+1+a} - \Phi_{i+a}}{\Phi_{i+1} - \Phi_{i}}.
\end{equation}
The indice $a$ is introduced to account for the flow direction: $a=-sign\left(u_{i+1/2}\right)$. 
The flux limiter function is then defined so that $\phi$ can, under certain conditions, yield upwind biased fluxes which are efficient to smoothen the 
solution where this is necessary and avoid the development of numerical oscillations. The values reached by the flux limiter are bounded so that $\phi$ possesses certain 
properties (the method must be TVD or Total Variation Diminishing, which guarantees that no local extremum is generated). One popular approach is given by the MC limiter:
\begin{equation}
\phi_{i+1/2} = max\left(0,min\left(2r_{i+1/2},0.5\left(1+r_{i+1/2}\right),2\right)\right).
\label{MClimiter}
\end{equation}
Replacing relation \ref{MClimiter} into \ref{fllw}, one finds the exact expression for the flux limited Lax-Wendroff method. In the model, several flux limiter functions were
coded. The choice of the method is however hard coded and this is the minmod limiter that is used by default (this is the most dissipative limiter possible, thus limiting the
appearance of spurious oscillations).

In practice, evaluating the limiter $\phi$ in each grid cell of the domain appears to be quite CPU expensive, especially considering that a large part of the domain may actually
not exhibit strong gradients. As a way to improve the method, Durran \cite{Dur2010} proposes to use a criterion $\lambda$ inspired by the WENO schemes (see below) to determine 
regions where flux limiting is necessary. By doing so, it is possible not only to improve the code's efficiency but also to improve
the overall accuracy of the scheme as the damping provided by the limiter will only be applied where necessary. Durran defines a first set of positive quantities:
\begin{equation}
\gamma_{l} = \left(\Phi_{i+l} - \Phi_{i-1+l}\right)^{2} + \left(\Phi_{i-1+l} - \Phi_{i-2+l}\right)^{2},
\end{equation}
with $l$ varying from $0$ to $2$ in the present case. From there, the criterion $\lambda$ is defined by:
\begin{equation}
\lambda_{i+1/2} = \frac{max\left(\gamma\right)}{min\left(\gamma\right) + \epsilon}.
\end{equation}
The flux limiter will be calculated and applied only when $\lambda$ exceeds a threshold value set to $10$ (the value can be modified in {\it shared\_dat.f90}).

\subsubsection{3rd order TVD flux limited scheme}

The 3rd order TVD method relies on the same principles as the Lax-Wendroff scheme introduced previously. The only difference is that the high-order flux $F_{i+1/2}^{HO}$ is
now defined using the so-called $\kappa=1/3$ scheme:
\begin{equation}
F_{i+1/2}^{HO} = u_{i+1/2}\left[\Phi_{i} + \frac{1-\kappa}{4}\left(\Phi_{i}-\Phi_{i-1}\right) + \frac{1+\kappa}{4}\left(\Phi_{i+1}-\Phi_{i}\right)\right].
\end{equation}
The coefficient $\kappa$ varies between 0 and 1, but for a value of $1/3$, a 3rd order upwind biased approximation is retrieved. 
The above relation is only true for a positive velocity. The flux associated to a negative velocity can be found by taking the exact symmetric of the above with respect to $i+1/2$.
Using the upstream method as the low-order monotone scheme and recasting the 3rd order method into a flux-limited form, we can find the following simple interpolation of $\phi$ at
$i+1/2$:
\begin{equation}
\Phi_{i+1/2} = \Phi_{i} + \frac{1}{2}\phi_{i+1/2}\left(\Phi_{i} - \Phi_{i-1}\right).
\end{equation}
Let's note that if the flow velocity is negative, the equivalent relation will be written:
\begin{equation}
\Phi_{i+1/2} = \Phi_{i+1} - \frac{1}{2}\phi_{i+1/2}\left(\Phi_{i+2} - \Phi_{i+1}\right).
\end{equation}

Following Hundsdorfer et al. \cite{Hal1995}, the flux limiter will include the $\kappa$ coefficient as $\phi$ will be defined as a function of $K_{i+1/2}$:
\begin{equation}
K_{i+1/2} = \frac{1 - \kappa}{2} + \frac{1 + \kappa}{2} r_{i+1/2}.
\end{equation}
The slope ratio $r$ is still defined as the ratio between two successive gradients. Finally, $K$ must be introduced into the flux limiter formulation to obtain $\phi$ for the $\kappa=1/3$ scheme:
\begin{equation}
\phi_{i+1/2} = max\left(0,min\left(2r_{i+1/2},0.5\left(1+K_{i+1/2}\right),2\right)\right).
\end{equation}

In order to improve the method, the same selective procedure as the one proposed for the Lax-Wendroff scheme can be applied. $\lambda$ is defined exactly in the same way in both
cases. 

\subsubsection{3rd order MUSCL scheme}
The MUSCL scheme \cite{KT2000} is a central method so that it generates minor numerical dissipation which is a great advantage compared to the usual upwind biased flux limited 
methods. We start from relation \ref{flux0} and express each flux in the difference as the average between a left and right flux interpolation at the cell face:
\begin{equation}
F_{i+1/2} = \frac{1}{2}\left(F_{i+1/2}\left(\Phi^{R}\right) + F_{i+1/2}\left(\Phi^{L}\right)\right).
\label{flux_muscl}
\end{equation}
The $*$ symbol denotes here the MUSCL reconstructed flux and $R$ and $L$ are respectively the right and left interpolated fluxes.
These right and left fluxes can be expressed in various manners, and we choose here to use 3rd order polynomial reconstructions using a 3 point stencil extending from
$i-1$ to $i+1$ for the left flux and $i$ to $i+2$ for the right flux. $F^{R}$ and $F^{L}$ are respectively functions of $\Phi^{R}$ and $\Phi^{L}$ the interpolated states
on the left and right sides of the face. Using the 3rd order ENO reconstructions we write:
\begin{eqnarray}
\Phi_{i+1/2}^{L} &=& -\frac{1}{6}\Phi_{i-1} + \frac{5}{6}\Phi_{i} + \frac{1}{3}\Phi_{i+1} \label{phiL} \\
\Phi_{i+1/2}^{R} &=& \frac{1}{3}\Phi_{i} + \frac{5}{6}\Phi_{i+1} - \frac{1}{6}\Phi_{i+2}. \label{phiR}
\end{eqnarray}
Further informations on ENO reconstructions and a table of the various coefficients are available for example in Durran \cite{Dur2010}.
The principle of the MUSCL method consists in rewriting these interpolated scalar quantities by making use of flux limiters. Rearranging \ref{phiL} and \ref{phiR}
into a flux limited form, both expressions become:
\begin{eqnarray}
\Phi_{i+1/2}^{L} &=& \Phi_{i} + \frac{1}{2}\phi_{i+1/2}\left[\frac{1}{3}\left(\Phi_{i} - \Phi_{i-1}\right) + \frac{2}{3}\left(\Phi_{i+1} - \Phi_{i}\right)\right] \\
\Phi_{i+1/2}^{R} &=& \Phi_{i+1} - \frac{1}{2}\phi_{i+3/2}\left[\frac{1}{3}\left(\Phi_{i+2} - \Phi_{i+1}\right) + \frac{2}{3}\left(\Phi_{i+1} - \Phi_{i}\right)\right],
\end{eqnarray}
where $\phi$ is the flux limiter function defined previously.
When $\phi=1$, which is expected in smooth parts of the flow field, $\Phi^{L}$ and $\Phi^{R}$ reduce to the 3rd order ENO interpolations so that the total reconstructed 
flux $F^{*}$ given by \ref{flux_muscl} yields a 4th order central method. On the opposite, when $\phi=0$, reconstructed fluxes $F^{*}$ are approximated by 2nd order central
differences. Finally, using the reconstructed scalars at the faces, we compute the advective fluxes following:
\begin{equation}
F_{i+1/2}^{*} = u_{i} \Phi^{*}_{i+1/2},
\label{flux0}
\end{equation}
where no velocity interpolation at the face is required because of the staggered grid and with $\Phi^{*}_{i+1/2}$ calculated following relation \ref{flux_muscl}.

Where stronger gradients are present, a more severe upwinding is necessary. However, due to the fact that the wind speed may change sign within the domain, upwinding 
must be done properly accounting for wind direction. Using MUSCL reconstruction \ref{flux_muscl}, Kurganov and Tadmor \cite{KT2000} proposed an upwind biased scheme based 
on the simple Lax-Friedrichs flux splitting which will provide sufficient numerical dissipation. We then write: 
\begin{equation}
F^{*}_{i+1/2} = \frac{1}{2}\left(F_{i+1/2}^{R} + F_{i+1/2}^{L}\right) - a_{i+1/2}\left(\Phi^{R}_{i+1/2} - \Phi_{i+1/2}^{L}\right).
\label{flux_LF}
\end{equation}
The velocity $a_{i+1/2}$ can be interpreted in the present case as a CFL based criterion and is defined by: 
$a_{i+1/2} = \max\left(\left|u_{i}\right|,\left|u_{i+1}\right|\right)$. This simplification is possible because the flow is non-divergent. This upwinding method is only
applied in the vertical direction where the strongest gradients are expected. In the horizontal plane, the original central method is retained in order to limit 
numerical dissipation. 


%In the flux limited method, we seek for fluxes of the form (Durran \cite{Dur2010}):
%\begin{equation}
%F_{i+1/2} = F^{L}_{i+1/2} + \frac{1}{2}K_{i+1/2}\left(F^{H}_{i+1/2} - F^{L}_{i+1/2}\right),
%\label{flux1}
%\end{equation}
%where $K$ is a function to be defined later and $F^{L}$ and $F^{H}$ are low and high order fluxes respectively. As the flow velocity is 
%expected to be either positive or negative, the upstream method is chosen as the low-order scheme:
%\begin{equation}
%F^{L}_{i+1/2} = \frac{u_{i}+\left|u_{i}\right|}{2} \Phi_{i+1} + \frac{u_{i}-\left|u_{i}\right|}{2} \Phi_{i}.
%\label{upstream}
%\end{equation}
%The high order method is based on 4th order or 2nd order central finite differences (similar as momentum advection):
%\begin{equation}
%F^{H}_{i+1/2} = u_{i}\left[\Phi_{i+1} + \frac{1}{2}\left(\Phi_{i+1} - \Phi_{i}\right) + C_{1}\left(\Phi_{i} - \Phi_{i-1}\right) - 
%C_{1}\left(\Phi_{i+2} - \Phi_{i+1}\right)\right],
%\label{4thorder}
%\end{equation}
%with $C_{1}=0.$ for the 2nd order scheme and $C_{1}=1/12$ to achieve 4th order accuracy. Note that as the velocity is shifted by half a grid point compared to the scalar
%location, no velocity interpolation is required. Introducing these two expressions in \ref{flux1} and rearranging the terms yields:
%\begin{equation}
%F_{i+1/2} = F_{i+1/2}^{L} + \frac{\left|u_{i}\right|}{2}K_{i+1/2}\left(\Phi_{i+1} - \Phi_{i}\right),
%\label{flux2}
%\end{equation}
%where the upstream flux is still explicitly employed, and with $K$ being defined by:
%\begin{equation}
%K_{i+1/2} = \frac{1}{6}\left(r_{i+1/2} - s_{i+1/2}\right) + 1,
%\end{equation}
%for the 4th order scheme and:
%\begin{equation}
%K_{i+1/2} = r_{i+1/2},
%\end{equation}
%for the second order. The so-called slope ratios $r$ and $s$ represent the ratios between two successive gradients:
%\begin{eqnarray}
%r_{i+1/2} &=& \frac{\Phi_{i+1+a} - \Phi_{i+a}}{\Phi_{i+1} - \Phi_{i}} \\
%s_{i+1/2} &=& \frac{\Phi_{i+1-a} - \Phi_{i-a}}{\Phi_{i+1+a} - \Phi_{i+a}},
%\end{eqnarray}
%with $a=-sign\left(u_{i}\right)$ taking the value $1$ or $-1$ depending on the direction of the flow.
%At the moment, no limitation has been imposed and relation \ref{flux2} is equivalent to central finite differences when developed. The limitation is further
%imposed on $K$ so that:
%\begin{equation}
%F_{i+1/2} = F_{i+1/2}^{L} + \frac{\left|u_{i}\right|}{2}\phi_{i+1/2}\left(\Phi_{i+1} - \Phi_{i}\right),
%\label{flux3}
%\end{equation}
%with $\phi_{i+1/2}$ the limited slope ratio, function of $K_{i+1/2}$. When $\phi=0$, which we can expect to occur in strong gradient regions, the low-order scheme is 
%used and a smooth solution must be retrieved, free of spurious numerical oscillations. On the opposite, when $\phi=1$ a high-order solution is obtained. 

\subsubsection{3rd order WENO scheme}
Another high resolution method, which provides a little bit more dissipation than the MUSCL scheme, is based on WENO reconstructions. Following the approach described
by Jiang and Shu \cite{JS1996}, a 3rd order WENO method can be constructed by first defining 2nd order scalar interpolations at the cell faces on two successive stencils:
\begin{eqnarray}
\Phi_{i+1/2}^{1} &=& -\frac{1}{3}\Phi_{i-1} + \frac{2}{3}\Phi_{i} \\
\Phi_{i+1/2}^{2} &=& \frac{1}{2}\Phi_{i} + \frac{1}{2}\Phi_{i+1}.
\end{eqnarray}
The scalar advective flux at the face is then evaluated using a weighted combination of these two interpolations:
\begin{equation}
F_{i+1/2} = \left(\omega_{1}\Phi_{i+1/2}^{1} + \omega_{2}\Phi_{i+1/2}^{2}\right) u_{i}.
\end{equation}
The weights $\omega_{1}$ and $\omega_{2}$ are then computed as follows:
\begin{equation}
\omega_{k} = \frac{\alpha_{k}}{\alpha_{1} + \alpha_{2}}, \hspace{2cm} \alpha_{k} = \frac{C_{k}}{\left(\beta_{k} + \epsilon\right)^{2}},
\end{equation}
with $k$ equals to 1 and 2, $\epsilon$ a small quantity (typically equals to $10^{-6}$) to avoid division by 0, $C_{1}=1/3$ and $C_{2}=2/3$ the optimal weights and
$\beta_{k}$ smoothness functions that indicate whether steep gradients are located in the vicinity of the cell. The smoothness functions are given by:
\begin{equation}
\beta_{1} = \left(\Phi_{i} - \Phi_{i-1}\right)^{2} \hspace{1cm} \mbox{and} \hspace{1cm} \beta_{2} = \left(\Phi_{i+1} - \Phi_{i}\right)^{2}.
\end{equation}
The above describes the simplest form of the 3rd order WENO method, for a flow possessing a positive velocity.  

In a similar fashion as for the MUSCL method, we must account for the fact that wind may change direction within the domain thus requiring a different upwinding. If the
flow velocity is negative, we must then evaluate the fluxes as the exact symmetry of those described above with respect to the point $x_{i+1/2}$. As a result, we use:
\begin{eqnarray}
\Phi_{i+1/2}^{1} &=& \frac{1}{2}\Phi_{i} + \frac{1}{2}\Phi_{i+1} \\
\Phi_{i+1/2}^{2} &=& \frac{2}{3}\Phi_{i+1} - \frac{1}{3}\Phi_{i+2},
\end{eqnarray}
that are weighted in a similar manner as previously. At each cell face, we can hence define a left and right WENO reconstruction (recalling the MUSCL method) that can be 
combined through the Lax-Friedrichs flux splitting \ref{flux_LF} to yield the correct upwinding.

The WENO schemes are known to be rather dissipative also in regions where only weak gradients are present. Several methods were proposed in order to correct this 
behaviour and select more properly the regions where the weighting procedure should be applied (this can be done f.i. by changing the formulation of the $\beta_{k}$
coefficients). In the MIMICA model, we chose to use a smoothness indicator $\lambda$ that will allow to identify the disconinuities in the scalar field. The WENO procedure
is then used only in regions where $\lambda$ is greated than a given threshold $\lambda_{max}$, whereas the $\beta_{k}$ coefficients are set to $0$ elsewhere. The
strong indicator from Xu and Shu \cite{XS2005} was selected here. If we define the coefficients $\beta_{i} = \left(\Phi_{i} - \Phi_{i-1}\right)^{2} + \epsilon$, we can 
the construct the indicator as follows:
\begin{equation}
\lambda_{i} = \frac{\alpha_{i}}{\alpha_{i} + \gamma_{i}},
\end{equation}
with:
\begin{equation}
\alpha_{i} = \left(\frac{\beta_{i}}{\beta_{i-1}} + \frac{\beta{i+1}}{\beta{i+2}}\right)^{2} \mbox{\hspace{1cm} and \hspace{1cm}} 
\gamma_{i}�= \frac{\left(\Phi_{max} - \Phi_{min}\right)^{2}}{\beta_{i}}.
\end{equation}
In the above, $\epsilon$ is a small constant to avoid divisions by 0 (we take $\epsilon = 10^{-6}$). Following this definition, the indicator is a normalized quantitity 
that will take a value close to 0 in smooth region and tends to unity close to strong gradients when $\alpha_{i} >> \gamma_{i}$. The threshold value $\lambda_{max}$ is set
by default to $0.005$. The main advantage with using this definition is that $\lambda$ is defined as varying between 0 and 1 so that the threshold value can be set
independently on the considered scalar (several orders of magnitude separate the largest and smallest transported scalars so that it can be hard to properly identify
discontinuities with a single indicator and threshold).

\subsection{Multidimensional advection}
Untill that point, we have only considered one dimensional scalar advection. In multiple dimensions, the direct application of flux limited methods can cause some 
implementation difficulties so that an explicit direction splitting method was employed in this code. Scalars are hence transported successively along each direction 
before the total fluxes are reconstructed. In three dimensions, four successive steps are thus required:
\begin{eqnarray}
\bullet & \Phi_{i,j,k}^{1} & = \Phi_{i,j,k}^{n} + \Delta t \frac{F_{i+1/2,j,k}\left(\Phi_{i,j,k}^{n}\right) - F_{i-1/2,j,k}\left(\Phi_{i,j,k}^{n}\right)}{\Delta x_{i}} \\
\bullet & \Phi_{i,j,k}^{2} & = \Phi_{i,j,k}^{1} + \Delta t \frac{F_{i,j+1/2,k}\left(\Phi_{i,j,k}^{n}\right) - F_{i,j-1/2,k}\left(\Phi_{i,j,k}^{n}\right)}{\Delta x_{j}} \\
\bullet & \Phi_{i,j,k}^{3} & = \Phi_{i,j,k}^{2} + \Delta t \frac{F_{i,j,k+1/2}\left(\Phi_{i,j,k}^{n}\right) - F_{i,j,k-1/2}\left(\Phi_{i,j,k}^{n}\right)}{\Delta x_{k}} \\
\bullet & \mathcal{F}^{n}_{i,j,k} & = \frac{\Phi_{i,j,k}^{3} - \Phi_{i,j,k}^{n}}{\Delta t},
\end{eqnarray}
with $\mathcal{F}^{n}_{i,j,k}$ representing the total advective flux in a given grid cell for scalar $\Phi$. 

\subsection{Precipitation}
Considering any hydrometeor moment ($N$ or $Q$) denoted $\Psi$, the related precipitation flux reads:
\begin{equation}
P^{\Psi} \equiv -\frac{1}{\rho} \frac{\partial \rho \left<V_{p}\right> \Psi}{\partial z},
\end{equation}
with $\left<V_{p}\right>$ the mass weighted averaged terminal fall speed of the chosen hydrometeor kind. This velocity is expressed following relation \ref{TFS} and exhibits
a non-linear dependence on the two hydrometeor moments. The modeled precipitation flux $P^{\Psi}$ is thus a non-linear derivative term which can generate inherent
numerical instabilities in the vicinity of discontinuities or sharp gradients (this is a property of non-linear PDEs). In order to avoid the development of such
instabilities, precipitation fluxes are discretized using upstream 1st order differences which introduce large numerical dissipation damping any
numerical wave. We write (to simplify, only the vertical index $k$ is shown):
\begin{equation}
P^{\Psi}_{k} = \frac{1}{\rho_{k}} \frac{\rho_{k+1}\left<V_{p}\right>_{k+1} \Psi_{k+1} - \rho_{k}\left<V_{p}\right>_{k} \Psi_{k}}{\Delta z_{p}}.
\end{equation} 
Note that $\left<V_{p}\right>$ being an hydrometeor property, it is calculated at the scalar location and is thus cell-centered.

\subsection{The pressure solver}
\label{pressure_solver}

\subsubsection{Diagnostic pressure}
Compared to the original CRM model, the LES code solves for the purely incompressible equations. Two main reasons motivated this choice. First, by getting rid of acoustic
perturbations inside the domain, the stability criterion is fixed by the advective velocity instead of the acoustic one. This latter being of the order of $300$ m/s in the
atmosphere, the stability criterion is much more constraining in a compressible flow. Very short time steps or time-splitting procedures (not really efficient
and accurate) are therefore required in order to fulfill the stability requirements. The second reason is related to the advection discretization described previously. In 
the CRM version of the code, a pseudo-anelastic form of the equations are solved where a divergence free advection term is discretized and a pressure equation is solved 
jointly (the system is over-constrained). This type of method will be conservative only if the flow is actually divergence free, condition that is not properly respected 
if a pressure equation is solved. Forcing the flow to remain divergence free by the method presented below ensures that scalars and momentum are conserved when using the 
chosen advection schemes.

When the anelastic approximation is made, the velocity divergence vanishes and the continuity equation is no longer required. As a result, the perturbation Exner
pressure has to be calculated differently than by solving a specific time-dependent balance equation. Combining the momentum equation \ref{momentum} and the anelastic 
constraint \ref{divergence} yields an elliptic equation for the perturbation pressure that can easily be solved during each time step.

Let's take the divergence of the momentum equation \ref{momentum}. Given that the incompressible flow is divergence free, the divergence of the momentum tendency
vanishes and we obtain:
\begin{equation}
\frac{\partial}{\partial x_{i}}\left[-u_{j}\frac{\partial u_{i}}{\partial x_{j}} - Cp \theta_{00} \frac{\partial \pi}{\partial x_{i}} + b\delta_{i,3} + f_{k} \left(u_{j}-u_{gj}\right)\epsilon_{i,j,k} + \frac{1}{\rho}
\frac{\partial {\mathbf \tau^{u}}}{\partial x_{j}}\right] = 0.
\end{equation}
An equation for the perturbation pressure is then easily derived:
\begin{equation}
\frac{\partial^{2} \pi}{\partial x_{i}^{2}} = \frac{1}{Cp\theta_{00}} \frac{\partial}{\partial x_{i}}\left[-u_{j}\frac{\partial u_{i}}{\partial x_{j}} +
b\delta_{i,3} + f_{k} \left(u_{j}-u_{gj}\right)\epsilon_{i,j,k} + \frac{1}{\rho}
\frac{\partial {\mathbf \tau^{u}}}{\partial x_{j}}\right].
\label{pressure}
\end{equation}

To solve this elliptic equation, the pressure solver uses a mixed method with FFTs in the horizontal plane and the inversion of a tridiagonal system in
the vertical direection. Assuming that the lateral boundaries are periodic (which is a prerequisite for FFT in the horizontal plane), it is possible to transform  
equation \ref{pressure} into the Fourier space where second order derivatives become simple multiplications:
\begin{equation}
k^{2}\hat{\pi} + l^{2}\hat{\pi} + \frac{\partial^{2} \hat{\pi}}{\partial z^{2}} = \hat{{\mathcal F}},
\end{equation}
where $\hat{.}$ denotes the Fourier transform of a given variable and $k$ and $l$ are the wave numbers in the $x$ and $y$ directions respectively. Approaching the 
second order derivatives along $z$ by central differencies yields a tridiagonal system which can be efficiently resolved using standard methods. Transforming the 
solution $\hat{\pi}$ back into the physical space gives a very accurate estimation of the perturbation pressure.

Various methods can be employed to solve the coupled pressure/velocity system within one time-step: a simple explicit method was chosen in MIMICA. 
The pressure term appearing in the momentum equation is not added during the first integration step. Instead, we calculate an intermediate value of the
velocity vector at time $n+1$, denoted ${\mathbf u^{*}}$, with $\frac{\partial u_{i}^{*}}{\partial x_{i}} \neq 0$. The momentum tendency divergence is computed during 
this step. At the end of the time step, equation \ref{pressure} is solved to yield a diagnostic pressure which ensures a divergence free flow. This new pressure at 
time $n+1$ is then used to correct the intermediate velocity field ${\mathbf u^{*}}$ and find the final velocity field at $n+1$. The complete procedure is summarized
below, where Euler forward is used for time integration:
\begin{eqnarray}
\frac{u_{i}^{*} - u_{i}^{n}}{\Delta t} &=& {\mathcal F}^{n}_{i} \mbox{\hspace{2cm} with } \frac{\partial u_{i}^{*}}{\partial x_{i}}\neq 0 \\
\frac{\partial^{2} \pi^{n+1}}{\partial x_{i}^{2}} &=& \frac{1}{Cp \theta_{00}}\frac{\partial {\mathcal F}^{n}_{i}}{\partial x_{i}} \\
\frac{u_{i}^{n+1} - u_{i}^{*}}{\Delta t} &=& - Cp \theta_{00} \frac{\partial \pi^{n+1}}{\partial x_{i}}
\end{eqnarray}

\subsubsection{Prognostic pressure}
The MIMICA model must be used both as a CRM and as a LES model. As a result, it was decided that although the diagnostic pressure method is far more efficient, a
prognostic pressure option should be kept in order to keep the main CRM features intact. 

When the prognostic pressure option is turned on (PRESS\_DIAG is FALSE in {\it start}), an equation for the dynamic pressure perturbation is solved along with the other
governing equations. Following Klemp and Wilhelmsson, an equation for the dynamic pressure can easily be derived and after further simplifications, the following equation
is obtained:
\begin{equation}
\frac{\partial \pi}{\partial t} = - \frac{c^{2}}{\rho_{0} Cp \theta_{v0}^{2}}\frac{\partial \rho_{0} \theta_{v0} u_{i}}{\partial x_{i}},
\end{equation}
where the subscript $0$ denotes the initial reference state and $c^{2} = \sqrt{\gamma r T}$ is the sound speed in the atmosphere ($\gamma$ is the ratio between the
constant pressure and constant volume heat capacities and is set to $1.4$ in the atmosphere). 

Although written in a quite simple form, the above pressure equation introduces numerical difficulties that require the introduction of temporal substeps. The pressure
equation coupled to the momentum equations allow acoustic waves to propagate within the domain. As a result, numerical stability can only be achieved if these acoustic
waves are properly temporally resolved (see next section). Considering the acoustic CFL criterion must be verified to ensure stability, and knowing that the speed of
sound in the atmosphere is of the order of 300 m/s, it comes that $\Delta t$ must be reduced by a factor close to 300 to maintain stability compared to a similar
simulation with diagnostic pressure.

The prognostic pressure formulation is therefore quite time consuming, unless specific methods are used which reduce considerably the total CPU time for each step. We will
consider that all the other physical processes evolve with characteristic time scales much larger that the acoustic waves and are sufficiently well resolved by the large 
convective time steps. The use of the small acoustic time steps can therefore be limited to the resolution of the coupled velocity/pressure system. Following Klemp and
Wilhelmsson we introduce a time splitting method where a small enough time step is used to solve the above pressure equation as well as the momentum equations whereas all
the other physical processes, including the transport of all the other scalars, are solved using the original convective time step. Subiterations are then performed for
which only the advective terms in the pressure and momentum equations as well as the pressure term in the momentum equations are updated each time. 

The validity of both methods in a given case is usually hard to assess. The diagnostic pressure method introduces a large dependence of the results on the initial field 
as the Poisson solver needs to assume an initial pressure solution to solve the problem. This may lead to results converging towards a different stationnary state than
expected. However, in the prognostic pressure method case, the pressure equation presented above is in fact simplified and normally contains several extra terms. These
simplifications also introduce some kind of dependency of the results on initial assumptions, so that the computed converged field may also be erroneous. Both methods can
thus be viewed as equivalent in terms of accuracy but not in terms of CPU efficiency: the subiterations introduced in the prognostic pressure case dramatically reduce the
model's performances.

\subsection{Criterion for numerical stability}

Time integration performed in the MIMICA model uses only explicit schemes and are therefore conditionally stable. This means that to ensure numerical stability throughout the
computation and avoid the development of spurious oscillations, the time-step used for temporal integration must remain within prescribed bounds. 

Without giving complicated details, spectral analysis of the time integration scheme leads to the elaboration of the so-called CFL criterion which is usually expressed under 
the form:
\begin{equation}
\left(\frac{\left|U\right|}{\Delta L} + N\right)\Delta t < CFL_{max}.
\label{CFL}
\end{equation}
In the above condition, gravity waves were considered so that $N$, the Brunt-Vaisala frequency, appears as an additional term in the CFL number. The velocity $\left|U\right|$
and $\Delta L$ are obviously chosen in one-dimensional cases, but when multidimensional simulations are performed, several cases can be considered and the proper definition of
$U$ is no longer trivial. Using a staggered grid system actually simplifies the problem and enables to somewhat reduce the limitation imposed by the CFL number.
As a matter of fact, in such a case, it can be shown that the velocity used for the criterion reduces to (here in 3D):
\begin{equation}
\frac{\left|U\right|}{\Delta L} = \max\left(\frac{\left|u\right|}{\Delta x},\frac{\left|v\right|}{\Delta y},\frac{\left|w\right|}{\Delta z}\right)
\end{equation}

In \ref{CFL}, $CFL_{max}$ is a threshold over
which the numerical scheme starts to get unstable. Its value depends both on the integration scheme as well as on the choice of method made for wind and scalar advection.

In MIMICA, the recommended typical values for $CFL_{max}$ depending on the integration scheme are: for Leap-Frog time differencing, 
we usually take $CFL_{max} \approx 0.7$, for the Runge-Kutta 3 method, $CFL_{max} \approx 1.33$ and for the Runge-Kutta 44, $CFL_{max} > 2$. The values indicated here 
were determined experimentally using at least 4th order central schemes for advection. They may thus differ significantly from theoretical criteria often found in
the litterature and corresponding to ideal cases. 

In practice, $CFL_{max}$ can be prescribed by the user in {\it cm.nml}. We recommend to choose values of the threshold slightly lower than the one suggested above. The CFL
criterion is used to determine the maximum time-step allowed during the simulation. If $CFL_{max}$ is exceeded at at least one particular location inside the domain, the
time-step is automatically decreased to verify relation \ref{CFL} (done in {\it check.f90}).

When pressure is solved in a prognostic manner, that is when a pseudo-anelastic set of equations is solved, sound waves are allowed to propagate inside the numerical domain
and this is reflected through the CFL criterion. A spectral analysis can show that the stability criterion for the numerical scheme reduces in that case to:
\begin{equation}
\frac{\left|U + c\right| \Delta t}{\Delta x} < CFL_{max},
\end{equation}
where $c$ is the sound celerity. Given the fact that $c$ is usually very large compared to $U$ (of the order of 300 m/s), and assuming that the $N$ term arising from gravitational 
waves can be neglected, it appears that the CFL criterion in presence of acoustic waves is much more contraining than in a fully compressible case.

When considering the CFL number as a way to assess numerical stability of a scheme, one has to keep in mind that this kind of criterion has been originally derived from one-dimensional
problems in ideal cases. Stability may thus be much more sensitive to other parameters than just the CFL number (for example the geometry or the advection schemes). As an example, Durran 
\cite{Dur2010} shows that although combining a 4th order central finite-difference advection scheme with Leapfrog time differencing should yield a stability criterion of $2$, it is actually 
observed in practice that the CFL should not exceed $0.73$. As a result, the CFL method described above to limit the time step provides only a very simplistic solution which may not be true 
for the values of $CFL_{max}$ recommended. 

\subsection{MPI communications and domain decomposition}

The LES model is parallelized using the Message Passing Interface MPI. This requires the decomposition of the numerical domain in a given number of elements within which
the governing equations will be solved by a single CPU. Parallelization requires the addition of ghost cells on the sides of each local domain. The informations 
contained in these cells must then be communicated between neighbouring domains in order to allow a proper calculation of gradients, even at the internal boundaries. 
At the moment, domain decomposition can be performed along the two horizontal directions using the {\it DECOMP\_2D} switch in {\it start} (if one wants 2D decomposition,
otherwise, default is decomposition along x only). This represents a major improvement compared to the original CRM model as 2D domain decomposition allows the use of more
CPUs at the same time thus increasing the efficiency of the code. 

When performing a parallel run, one must be aware of some limitations which have to be kept in mind:
\begin{itemize}
\item[$\bullet$] the domains created after decomposition all contain the exact same amount of internal points. The total number of cells prescribed in 
{\it start} must then be a multiple of the number of processors/domains;

\item[$\bullet$] as suggested at the end of the manual for output files, some of the MPI communications done are unnecessary and lead to an excessive increase of
the CPU cost: f.i. before each output operation, all the data are gathered on a single processor before being written;

\item[$\bullet$] the use of a spectral method in the horizontal plane to solve the Poisson equation (diagnostic pressure) has to be performed on a single domain/processor 
as the Fourier transforms are hard to parallelize; this requires that the momentum tendencies have first to be gathered on a single processor and that the resulting
diagnostic pressure has to be broadcast to all the processor after the calculation;

\item[$\bullet$] more generally, the implementation of MPI communications may not be properly optimized.
\end{itemize}

A number of communication modules and interfaces have been created to simplify the use of MPI communications. Thereby, instead of calling directly the MPI routines, it is 
often more appropirate to call the corresponding internal interface: {\it collect} allows to gather the data located on all the CPUs on the master processor, 
{\it distribute} performs the opposite operation by sending informations located on one processor to all the CPUs, {\it exchange} communicates the data contained in the 
ghost cells of two neighbouring domains (treatment of internal and lateral periodic boundaries) and {\it distexch} combines {\it distribute} and {\it exchange}.

\newpage
\section{Input/Output}

\subsection{{\it cm.nml} and {\it out.nml}}

The namelist file {\it cm.nml} (situated in the working directory) contains some model parameters and configurations that can possibly be changed from one case to another 
or before a restart. (some additional important parameters that are presently hard-coded should be included in {\it cm.nml}).
The file starts with the {\it \$cm\_in} line and ends with {\it \$END}. All the lines below are considered as comments. All the parameters appearing in the file are
defined and stored during the run in the {\it shared\_data} module, but they also need to be defined in the header section of the main {\it mimica.f90} routine as belonging
to the {\it cm\_in} namelist. The reading of this file is treated automatically as a namelist where the values indicated are affected to the corresponding
variable name. The order of the entries in the namelist doesn't matter.

Table \ref{namelist} lists all the parameters appearing in {\it cm.nml} and their role in the model. The default values assigned to all these parameters are 
defined in {\it INCLUDE\/default.h}. For binary codes, the value 1 activates the process.

{\it out.nml} is a second namelist file present in INCLUDE that is used to prescribe which variables must be written in the output files (see the Output subsection). The
correspoding {\it ou\_in} namelist is defined in the header section of {\it mimica.f90} and must be used exactly as the cm namelist. 

\begin{table}[!h]
\begin{center}
\begin{tabular}{ccc}
\hspace{0.1cm} Variable name \hspace{0.1cm} & \hspace{0.1cm} Default value \hspace{0.1cm} &
\hspace{0.1cm} Meaning \hspace{0.1cm} \\
\hline
dt & 1 s & Main time step for dynamics/microphysics \\
ds & 1 s & Small time step for prognostic pressure \\
cfl\_max & 0.75 & Maximum allowed CFL (smaller than max. expected for stability) \\
cfl\_min & 0.5 & Minimum allowed CFL (time step is increased if CFL < cfl\_min) \\
ldtfix & 0 & If 1, the time step $dt$ is kept constant, CFL depdendent otherwise \\
nstart & 1 & Loop index of the first time step \\
nstop & 1 & Final simulation time (default = sec.) \\
nunit & 1 & Unit for nstop (1 = sec, 60 = min...) \\
iax & 10 & Output frequency in sec. for entire fields (if <0, no output) \\
iav & 10 & Output frequency in sec. for the restart file \\
its & 10 & Output frequency in sec. for time series \\
ipro & 10 & Output frequency in sec. for profiles \\
ires & 10 & Output frequency in sec. for restart files \\
iradx & -5 & Frequency for radiation solver (if <0, no call) \\
i\_creact & 1 & Frequency for chemistry calculation (if <0, no call) \\
isurf & 0 & Surface fluxes modelling (see Boundary conditions) \\
dx & 2000 m & Cell size in x \\
dy & 2000 m & Cell size in y \\
dz & 500 m & Reference cell size in z \\
z1; z2 & 500; 900 m & Refinement bounds in z \\
sratio & 1 & Refinement ratio in z \\
zdamp & 1000. & Altitude where sponge layer is applied \\
tdamp & 300. & Time scale for damping in sponge layer \\
u0shift & 0 & Translation velocity in x \\
v0shift & 0 & Translation velocity in y \\
pran & 0.33 & Turbulent Prandtl number (if <0, no SGS for scalars above 100m) \\
cdiff & 0 & Strength of artificial diffusion for momentum advection \\
lmicro & 0 & Microphysics level: 0=no microphysics, 1=warm micro only, 2=pristine ice micro, 3=ice with aggregates \\
lndrop & 1 & Turns on prognostic cloud droplet number concentration ($N_{c}=cst$ otherwise) \\
drizz & 1 & Activation of rain conversion \\
c\_sed & 0 & Activation of cloud droplet sedimentation \\
lvent & 0 & Activation of ventilation effects for codensation/deposition \\
lgrowth & 0 & Direct integration of diffusional growth (0=explicit, 1,2=pseudo-analytic) \\
lfreeze & 0 & Turns on production of ice \\
drmin & 70 & Separation diameter between cloud droplets and rain drops (in micro m) \\
shf & 20 & Sensible heat flux at the surface \\
lhf & 80 & Latent heat flux at the surface \\
sst & 295 & Skin surface temperature \\
ssm & 0. & Surface moisture (if 0, ssm calculated from sst at 0 supersaturation) \\
zrough & 1.e-3 & Surface roughness height \\
alb & 0.6 & Surface albedo for radiation solver \\
Ddiv & 3.75e-6 & Large scale horizontal divergence forcing \\
nsmooth & 1 & Temporal smoothing of the solution (0 by default in not Leap-Frog) \\
ksmooth & 0.2 & Smoothing parameter in Asselin's filter \\
xnc0\_c & 50.e6 & $N_{CCN0}$ parameter in CCN activation power law \\
xnc0\_k & 0.7 & $\kappa$ parameter in CCN activation power law \\
xn\_ccn0 & 50.e6 & Initial/reference CCN value \\
xn\_in0 & 100.e3 & Initial/reference IN value \\
\end{tabular}
\end{center}
\caption{{\it Model parameters defined in {\it cm.nml}.}}
\label{namelist}
\end{table}

\begin{table}[!h]
\begin{center}
\begin{tabular}{ccc}
\hspace{0.1cm} Variable name \hspace{0.1cm} & \hspace{0.1cm} Default value \hspace{0.1cm} &
\hspace{0.1cm} Meaning \hspace{0.1cm} \\
\hline
dpt & 1.e-4 & Maximum random perturbation amplitude on Pt \\
dqv & 1.e-5 & Maximum random perturbation amplitude on qv \\
if\_filter & 60 & Frequency for filtering after integration \\
nbulb & 1 & Number of initial warm bubbles \\
mx; my; mz & 375; 0; 6 & Initial poisition of the warm bubble \\
mx1; my1& 3; 1& Initial extent of the warm bubble \\
mx2; my2& 3; 1& Initial extent of the warm bubble \\
ioe; joe& 2; 1& Bubble size parameters ? \\
mz1& 1& Vertical position of the bubble \& first layer for random perturbations \\
mz2& 50& Vertical position of the bubble \& last layer for random perturbations \\
koe& 1& Bubble size parameter ? \\
j\_day & 67 & Day of the year \\
ctr\_lat & -4 & Latitude of the case (if <0, not relevant) \\
t\_utc & 0 & local time as compared to UTC \\
casename & xxxx & Case name \\
file\_xxxx & - & File names and locations 
\end{tabular}
\end{center}
\caption{{\it Model parameters defined in {\it cm.nml} (continued).}}
\label{namelist}
\end{table}

\subsection{{\it start}}

The {\it start} file contains switches (environement variables) that activate or deactivate certain modules, models or code features. It controls also the type of 
model executable created during compilation (regular, debug, parallel...). The file is closely related to {\it wheader.csh} which reads {\it start}, interpretes
the different environement variables and enables or disbales the compilation of the corresponding parts of the code. These parts can be easily identified within the 
model between the {\it \#ifdef} and {\it \#endif} statements. A list
of all the environement variables with their current status is generated after each compilation and written in {\it INCLUDE/ctrparam.h}. Note that
the switches control the compilation of these blocks, not their execution during the run: if a bug has been introduced in one of these blocks but the switch is not
turned on, the compiler won't produce any error. 

Table \ref{start} summarizes all the existing environement variables and which features they control. Regarding the choice of numerical method, only one environment
variable among all the ADV\_XXXX options as well as among LF, RK3 and RK4 must be set to TRUE at the same time. When They are all FALSE, the default time integration
scheme is the forward Euler method and scalar advection is performed using upwind biased finite differences. For the other parameters, a switch set to TRUE enables
the corresponding option.

\begin{table}[!h]
\begin{center}
\begin{tabular}{cc}
\hspace{0.3cm} Variable name \hspace{0.3cm} & \hspace{0.7 cm} Role \hspace{0.7cm} \\
\hline
NP & Number of processors/domains \\
CMPLER INTEL/PGI & Choice of intel or portland group compiler (default: gfortran) \\
R8 & Real type size (8 bits) \\
DEBUG & Builds a severe debug executable (recommended after heavy developments) \\
PROF & Builds an executable for profiling \\
SPMD & Builds an executable for parallel applications (with NP$\neq$1) \\
M3D & Three dimensional case if activated \\
DECOMP\_2D & Enables 2D domain decomposition for parallel runs \\
MAXX3D; MAXX2D & Total number of cells along x in 3D and 2D (with ghost cells at lateral boundaries) \\
MAXY3D; MAXY2D & Total number of cells along y in 3D and 2D (with ghost cells at lateral boundaries) \\
MAXZ3D; MAXZ2D & Total number of cells along z in 3D and 2D \\
NHYDRO & Total number of hydrometeors (different from nh, the number of modeled hydrometeors) \\
FINE & Allows grid refinement in the vertical direction \\
NEWRUN & Starts a completely new run if activated or restarts from previous results \\
AUTO\_INIT & Initializes the flow field using hard-coded functions if activated (reads in file otherwise) \\
CORIOLIS & Activates Coriolis forces \\
RADIA & Activates the radiation module \\
CHEM & Activates the chemistry module \\
AEROSOL & Activates the aerosol module \\
ADV\_AEROCHEM & Activates advection/mixing of chemistry and aerosol scalars \\
FORCE\_SUB & Assumes a forced large scale horizontal divergence if activated \\
SGS & Activates  \\
TKE & 1st order SGS closure with transported TKE \\
NUDGE & Nudging of the transported quantities \\
LSADV & Adding large scale scalar tendencies \\
MEANDATA & Outputs averaged results if true \\
FORM & Outputs results in formatted files if true \\
PROFILE & Outputs mean vertical profiles (horizontally averaged results) \\
PROF\_TS & Outputs time series of mean vertical profiles \\
OVERWRITE & Overwrites previous outputs if true: only the last output is kept \\
OUTPUTALL & Outputs one separate file after each output event \\
PRESS\_DIAG & Diagnostic pressure evaluation if true (prognostic otherwise) \\
ADV4 & 4th order advection scheme \\
ADV\_LW & Scalar advection using a 2nd order Lax-Wendroff limited scheme \\
ADV\_SMO & Scalar advection using Smolarkiewicz's MPDATA algorithm \\
ADV\_TVD & Scalar advection using a TVD 3rd order limited scheme \\
ADV\_MUSCL & Scalar advection using MUSCL 3rd order reconstruction \\
ADV\_WENO & Scalar advection using 3rd order WENO scheme \\
RK3 & 3rd order TVD Runge-Kutta  \\
RK4 & 4th order Runge-Kutta \\
LF & Leap-Frog integration method \\
SAT\_ADJ & Saturation adjustment method if true (no direct integration of diffusional growth) \\
SEIFERT & Uses Seifert \& Beheng warm microphysics \\
ICE\_NUC & Activates detailed ice nucleation scheme (not yet completed) \\
CCN\_ACTIV & Detailed CCN activation modelling \\
caseid & Case name 
 \end{tabular}
\end{center}
\caption{{\it Environement variables defined in {\it start}.}}
\label{start}
\end{table}
 
\subsection{Output files}

\subsubsection{Overview}
Outputs are controled by the {\it output.f90} routine. Four main types of output files are written: global outputs which contain the entire 2D/3D field for each selected
variable (see below), 1D vertical profiles where every selected quantities are averaged horizontaly, time series of these same vertical profiles and time series of scalars 
defined in {\it calmean.f90} and usually corresponding to domain extrema or domain ensemble averages. The frequence for these outputs is controled by {\it iax}, {\it its}
and {\it iprof} respectively. {\it its} controls at the same time the output of the scalar time series and 1D sounding time series. 

The total field files are simply given the same name as the output variable considered (not considering the possible extension notifying the output time). The 1D soundings
(instantaneous or time series) possess the name of the variable with the .pro extension. The scalar time series are all written in a single file named T\_S.
All these files are stored in the OUTPUT folder of the local directory. 

The selection of the output variables is done through the namelist {\it ou\_in} contained in {\it out.nml}. The namelist is defined in the exact same manner as 
{\it cm\_in} discussed in the previous section. {\it ou\_in} contains a list of logical variables out\_xxx  explicitly defining which quantities are to be 
written in the outputs. To select one variable, its corresponding logical switch has to be set to TRUE in {\it out.nml}. Two files can be created for each
selected quantity: one containing the whole fields and one for the vertical profiles (if the variable PROFILE is turned on). The names and unit numbers associated to
each output variable can be found in {\it INCLUDE/assign\_out.h}. Note that one file is created for each 2D/3D quantity,
even for multiple processor jobs: this implies that slow MPI communications are requested for each output variable and during each output step which
negatively affects the model performances.

\subsubsection{Output options}
Both 2D/3D fields and 1D profiles can be written either in formatted ASCII files or unformatted binary files. The choice is made via the FORM switch present in {\it start}
(the output will be written in ASCII if FORM is set to TRUE).
Binary files are less memory demanding and thus recommended but it is often convenient to be able to directly read the output files to track for instance the location of 
a singular point within the flow field. In addition, for these two types of outputs, it is possible to write either the instantaneous results or time averaged data 
depending on the status of MEANDATA in {\it start}. If time averaged data are written, the averages are calculated over a time lapse set in {\it cm.nml} by the parameter
{\it iav}. 

In the {\it start} file, the parameters OVERWRITE and OUTPUALL control the behavior of the model when several output times are met during the simulation. Three possibilities
can be chosen depending on the combinations between these two switches. When OUTPUTALL is set to TRUE, each of the different selected output variables (in {\it out.nml}) is
written in one separate file for each output event. The files contain the extension .xxxx, where xxxx is the physical time in seconds at which the file has been created.
This regards both the total 2D-3D fields and the 1D soundings. For the profile time series, only one single file for each output variable is created, independently on the
OUTPUTALL switch. This file possesses the extension .0000 to be easily identified. The OUTPUTALL option has the disadvantage to create a large amount of output files but it 
allows a simple post-treatment of these files. 

When OUTPUTALL is turned to FALSE, only one file is created for each variable at the beginning of the run. Depending on the value of OVERWRITE, these single
files can be either overwritten during each output (TRUE) or the new output can simply be written at the end of the file (if FALSE) so that all the different outputs are
still present within one file. 

\subsubsection{Post-processing}
A small post-processing tool has been developed in fortran: {\it post\_process.f90} to help processing the raw data files generated by MIMICA for a given number of selected
quantities (defined in {\it file\_list.dat}). The program merges all the different outputs and generates a single file, in a prescribed file format. When outputs were performed
at different simulation times, the tool can only handle the situation if one file have been created after each output (OUTPUTALL = TRUE). At the moment, the software can't read 
block output files written with the OVERWRITE switch turned to FALSE. 

The post\_process tool requires an input list {\it file\_list.inp} to select the files to be processed. This list is made of two parts: the first one, introduced by the 
keyword time, contains a list of times that we want to process. The figures written below must exactly correspond to the time extensions of the raw output files. If required, 
the profile time series can also be processed. In that case, the time list must also contain a line 0000 which is the extension for the profile time series files. The second 
part of the {\it file\_list.inp} file is introduced by the keyword name and contains a list of variable names to process. It is possible to comment one name line by adding
\# in the first column. Using the keywords, it is possible to select just a limited subset of results to process, at specific output times.

The post-processing software will create one single file for each output kind and for each selected time. The full fields will be written in {\it les\_outptu.tec} files, to
which is added an extension for the output time considered. The 1D soundings will be written in {\it profiles.dat} files (again with an extension specifying the time) whereas
the profile time series will be written in a single file named {\it profiles.2d.tec}. Except for the 1D profiles, all the other files are output in Tecplot format.
These files can be read directly by Tecplot but also by other visualization softwares such as ParaView. More formats can be added easily. 

\subsection{Restart files}

When the NEWRUN option in {\it start} is turned to FALSE, the MIMICA code can start a simulation from results obtained during a previous run. A restart file is required
for this operation, called {\it restart.dat} (in the folder DATA), which is automatically generated after each {\it ires} second or at the end of a successful run. 
{\it restart.dat} is a binary file where all the necessary quantities, that is all the prognostic and diagnostic results of the previous run, are gathered 
in a single data block. When NEWRUN is selected, the usual field initialization is skipped and {\it restart.dat} is read. 

For a restart simulation, a second file is required containing the initial soundings used to initialize the original run but also as reference states: PT0. This file is
created and employed in the same way as {\it restart.dat} but contains only the reference state under the form of 1D soundings. 

\newpage
\section{Main routines and data structures}
\label{structures}

\subsection{Main subroutines and overall hierarchy}

Without drawing the entire tree chart of the code, here is described the role played by some of the more important subroutines and their
interlinks with the rest of the model (we exclude all the shared and typedef modules as well as most of the routines used by
the detailed chemistry and aerosol modules). Three other public packages used by the code could also be added to this list: the LSODE solver, the Fast Fourier Transform 
package (fftpack) and the tridiagonal system solver. In addition, some elementary functions to calculate derived thermodynamical and microphysical properties can be found 
in the {\it typedef\_thermo} and {\it typedef\_hydro} structures. 

\begin{itemize}
\item[$\bullet$] {\bf mimica.f90} is the main routine of the program. Its only role is to define and read the input parameters stored in a common block and to provide 
an interface for {\it modelctl}.
\item[$\bullet$] {\bf gridno.F} is created automatically when the code is compiled. Its content is defined in the {\it wgrid.csh} script and consists mainly in some
fixed dimensions such as the domain size.
\item[$\bullet$] {\bf modelctl.f90} calls the subroutines in charge of the allocation of the data structures (defined as pointers in the parallel case), the 
initialization, the time-stepping (in {\it time\_step.f90}) and the outputs. It constitutes the main driving routine of the model. 
\item[$\bullet$] {\bf mpi\_prepare.f90} performs a certain number of preparatory tasks for MPI simulations: array and data structure allocation, communication of the
initial parameters to all the procs, MPI types definition and domain decomposition along 1 or 2 dimensions.
\item[$\bullet$] {\bf setrun.f90} initializes the simulation. Different options are available: the 2D or 3D fields can be initialized using prescribed soundings defined
at the end of {\it initial.f90} (stored in pt0, w0, p0.... in {\it shared\_data}) or by using output file produced by a previous run in {\it restart.f90}.
\item[$\bullet$] {\bf initial.f90} creates the initial fields. At the moment this can only be done through prescribed functions for all the necessary prognostic quantities.
The initialization is of course case dependent, see {\it init\_pro} at the end of the file for examples. The initial soundings can be perturbed in order to initiate natural 
convection either by creating a thermal bubble inside the domain or using weak random perturbations in the ptil and qv fields.
\item[$\bullet$] {\bf time\_step.f90} contains the interface for the choice of the numerical integration among 3 methods: 
{\it normal\_phys\_integ}, {\it rk4\_phys\_integ} and {\it forward}. The three solvers are built in the same way where the different stages of the integration
step are treated separately: first, the prognostic thermodynamic quantities are evaluated for the current time step by {\it cal\_thermo} (in {\it funcpack}), then momentum 
equations are solved by {\it udcomp} or {\it ud\_tq\_rk}, the scalar equations (for pt, qv and the hydrometeor moments) are solved during the next step by {\it tqcomp} 
or in {\it ud\_tq\_rk},and finally the pressure solver {\it pressure\_solver} is invoked to evaluate the diagnostic pressure and update the velocities.
\item[$\bullet$] {\bf ud.f90} integrates the momentum equations using the Leap-Frog method. The SGS term is first computed by {\it efud} (in {\it subgrid.f90}), and the 
advective tendencies are then evaluated by the routines contained in {\it windadv.f90}. The Coriolis forces are added if required and the buoyancy term for the vertical
velocity is calculated by {\it buoyancy} (in {\it funcpack}). The boundary conditions are then imposed via {\it vbc} and {\it lbc} before the
equations are integrated. Smoothing and damping at the top are provided if necessary. A prognostic equation for the TKE is also solved on the same model by
{\it tkecomp} if required. The divergence of the momentum tendencies used by the pressure solver are also calculated here by {\it gradtend}.
Note that the wind velocities wind1 and wind2 are not yet updated, the time step is completed only after the pressure solver. 
\item[$\bullet$] {\bf subgrid.f90} contains a package of subroutines used to calculate the sub-grid scale fluxes for all the variables. {\it kcomp} evaluates the eddy
diffusivity $K$ (Smagorinsky-Lilly model or prognostic equation for the TKE), and {\it efud} and {\it eff} evaluate the SGS term for momentum and scalars respectively.
\item[$\bullet$] {\bf sgspack.f90} calculates the SGS terms for all the quantities using the precalculations made in {\it subgrid.f90}. Calculates also the surface 
turbulent fluxes using either fixed heat fluxes or fixed surface temperature. 
\item[$\bullet$] {\bf funcpack.f90} is a package containing several useful functions called within the model. It includes for example the sponge layer damping, the Asselin 
filter and some functions used to calculate diagnostic properties (such as fluxes and horizontal averages).
\item[$\bullet$] {\bf integpack.f90} is a package used for time integration. It contains the RK3 and RK4 formula as well as the interface for the calls to
LSODE (although not used any more).
\item[$\bullet$] {\bf bc.f90} provides the different routines used to treat the vertical and lateral boundary conditions. 
\item[$\bullet$] {\bf tq.f90} solves governing equations for pt, qv and the hydrometeor moments following the same pattern as {\it udcomp} (Leap-Frog method).
The subroutines called are however specific to the treatment of scalars (f.i., scalar advection is calculated by {\it advq\_xxxx.f90}). There is no decoupling of dynamics and
microphysics so that all the tendencies are collected and integrated at the same time and all the scalars are updated at the end of the routine. 
\item[$\bullet$] {\bf advq\_xxx.f90} represent a set of routines performing scalar advection. xxxx must be replaced by the name of the schemes, namely lw, tvd, muscl, weno and smo.
\item[$\bullet$] {\bf ud\_tq\_rk.f90} integrates momentum and scalar equations at the same time following the Runge-Kutta 44-2S method. Because four subiterations are
required to complete the time-step, the momentum and scalars have to be integrated in parallel to ensure 4th order accuracy. Although this differs significantly from the ud 
and tq routines, the overall procedure for integration (calculation of the different contributions) remains the same. 
\item[$\bullet$] {\bf windadv.f90}, {\bf advq\_fd.f90}, {\bf advq\_lw.f90}, {\bf advq\_tvd.f90}, {\bf advq\_muscl.f90} and {\bf advq\_weno.f90} compute the advective fluxes for 
momentum and scalars respectively. For scalars, the different schemes are selected in {\it start}. 
\item[$\bullet$] {\bf microphysics1.f90} calculates the sources and sinks related to hydrometeor collision-coalescence processes, for both mass mixing ratios and number 
concentrations. The subroutines called (corresponding to each hydrometeor interaction) are included in {\it micropack1.f90}.
\item[$\bullet$] {\bf microphysics2.f90} calculates sources and sinks related to evaporation/condensation and sublimation/deposition. When the saturation adjustment procedure
is selected, rain drops and cloud droplet sources are not updated here. The analytic implicit approximation for supersaturation is also calculated here. All
the subroutines called are contained in {\it micropack2.f90} (including the saturation adjustment scheme). The sources calculated here are added to the overall tendencies for 
integration in tq.
\item[$\bullet$] {\bf micropack1.f90} and {\bf micropack2.f90} contain the subroutines evaluating each individual microphysical sources resulting from collision-coalescence
processes or phase transitions. They make use of several elementary functions defined in {\it shared\_hydro} and calculating some characteristic properties of each hydrometeor
distribution (like mean diameters).
\item[$\bullet$] {\bf micro\_diag.f90} contains all the "diagnostic" microphysics. This includes CCN activation, drop/droplet freezing (heterogeneous ice nucleation) and the
saturation adjustment scheme. 
\item[$\bullet$] {\bf micro\_ice.f90} calculates all the processes related to ice microphysics only and especially ice nucleation.
\item[$\bullet$] {\bf pressure\_solver.f90} contains all the routines needed to evaluate diagnostically the pressure and update the velocities after the time step.
The pressure solver calls the fast fourier transform code {\it fftpack.F} and the tridiagonal solver {\it tridiag.f90}, both defined externally. This routine
should be considered as a black box. 
\item[$\bullet$] {\bf grad.f90} calculates the pressure gradient term for momentum equations, the momentum tendency divergence for the pressure solver and the velocity
divergence. 
\item[$\bullet$] {\bf check.f90} checks the potential temperature as well as the CFL criterion after each time-step to determine if the simulation is stable. If not, either
the time-step is decreased so that the max CFL becomes lower than the one prescribed in {\it cm.nml} or the run is stopped. 
\item[$\bullet$] {\bf timevar.f90} updates time varying quantities. It is called once per timestep and used predefined functions to determine the values of parameters such as
surface fluxes and large scale scalar tendencies as functions of time.
\item[$\bullet$] {\bf calmean.f90} calculates the spatial and temporal averages that can be output if requested.
\item[$\bullet$] {\bf output.f90} controls all the possible outputs: instantaneous and mean fields, vertical profiles, time series... {\it outputpack.f90} contains the subroutines
called by output.
\item[$\bullet$] {\bf radiag.f90} is the main routine calling the radiation solver {\it radiatov.f90} or computing the case dependent radiative fluxes.
\item[$\bullet$] {\bf chemfront.f90} is the interface for the progress of the chemical concentrations. The advection of chemical species is performed by {\it advc\_m}.
\item[$\bullet$]�{\bf advcm.f90, advcsub.f90} are two routines used to transport the prognostic aerosols, chemical species or nuclei. Advection is performed using Bott's integral
scheme and SGS mixing using the same method as the other scalars. These two processes can be switched off by setting ADV\_AEROCHEM to FALSE in {\it start}. 
\item[$\bullet$] {\bf exchange.f90}, {\bf collect.f90}, {\bf distribute.f90} and {\bf distexch.f90}  provide the tools for MPI communications. 
\end{itemize}

\subsection{data structures}

The model uses several data structures for all the prognostic quantities as well as for some important diagnostic quantities. The use of data structures in fortran
has many advantages and contributes to the improvement of the code's readability. In the present case, data structures are invoked by calling {\it USE
shared\_xxxx} at the beginning of a routine, where {\it shared\_xxxx} is a module contained in the corresponding .f90 routine. They are made of two parts: the
type definition, called by {\it USE typedef\_xxxx}, containing the different elements of the data structure (f.i. in the wind structure, the type {\it wind}
includes the three components {\it u, v, w}), and the structure definition, where different names are declared and assigned to the given type. 

The structures are usually 2D or 3D arrays (4D for hydrometeors) where the three dimensional fields or 2D horizontal slices (for example at the surface) of major prognostic and 
diagnostic quantities are stored. All the structures defined in the code are saved and shared so that one can use 
the content of any given structure anywhere in the code simply by making use of the required shared module. The use of such structures limits possible errors related 
to bad array declarations/allocations. 

Note that the original fortran modules, containing the chemistry and aerosol properties, are used in different manner by some specific routines where array 
declarations are performed through the inclusion of .h files (the array names still need to be passed in the subroutine calls). This way of coding was generalized in the
CRM version of the model but should be changed in a near future to follow the new coding standards.

The data structures defined so far are described in the following table. The structure {\it shared\_data} is not referenced here. It contains the definition of
many constants, indices, input or dimensional parameters used throughout the model. Recently, the most common universal physical constants used throughout the code
were also defined in that file. As a result, physical constants should no longer explicitly appear in the model: they should all be defined first in {\it shared\_data.da}
and used directly through the use of this data structure.

Note that the library {\it typedef\_hydrometeor} defines several hydrometeor types, namely {\it hydrometeor}, {\it hydro\_second} and {\it hydro\_param} (the structure
names associated with these types are defined in {\it shared\_hydro}). Each new type has its own specific purpose: {\it hydrometeor} defines the pure hydrometeor
quantities transported by the model, that is the mass mixing ratio and number concentration of each kind; {\it hydro\_second} defines secondary hydrometeor quantities,
related to $n$ and $q$, but only necessary for microphysical process calculations ($\lambda$ parameter in the size distribution, the terminal fall speeds for precipitation
and the index $ifp$ locating the presence of each hydrometeor kind); finally, {\it hydro\_param} defines all the constants used for each kind in the microphysics, that is
for the mass-diameter law, for the fall speed-diameter law and for the size distribution. The definition of these three distinct hydrometeor types has led to an improved
readability of the microphysical module (especially for the collision processes) and more flexibility for the hydrometeor's laws (new hydrometeor kinds can be used simply
by changing the {\it hydro\_param} coefficients).

In addition, two shared modules, namely {\it shared\_hydro} and {\it shared\_thermo}, contain also elementary functions used to calculate some derived properties of the
given type. For example, {\it shared\_hydro} includes functions to evaluate the regular and mass weighted averaged terminal fall speeds, averaged hydrometeor diameters, 
the size distribution parameter $\lambda$ and several important constants. Note that these functions only use the constant parameters declared in {\it hydro\_param} so
that a change in the hydrometeor properties will be automatically taken into account in these calculations. {\it shared\_thermo} allows the calculation of the potential 
temperature, the absolute pressure, the saturation vapor pressures and mixing ratios and their derivatives with respect to temperature. These functions can be used anywhere 
in the model to evaluate these derived quantities as long as the corresponding module is in use. 

\begin{landscape}
\begin{table}[!h]
\begin{center}
\begin{tabular}{|c|c|c|c|c|}
\hline
\hspace{0.1cm} \hspace{0.1cm} Structure name \hspace{0.2cm} & \hspace{0.2cm} dimensions \hspace{0.5cm} & \hspace{0.5cm} elements \hspace{0.5cm} &
\hspace{0.5cm} meaning \hspace{0.5cm} & \hspace{0.5cm} purpose \hspace{0.5cm} \\
\hline
{\bf shared\_wind} & nux*nvy*nwz  & u (m/s) & wind vel. along x  & Contains the wind velocities. wind1 refers to the   \\
 &   & v (m/s) & wind vel. along y  & instantaneous winds at time $t$ ($t-1$ in  \\
 &   & w (m/s) & vertical wind vel. & LeapFrog), wind2 at time $t$ and wind0 at $t=0$ (initial) \\
\hline
{\bf shared\_state} & nx*ny*nz  & p (-) & Exner perturbation pressure & Prognostic and diagnostic state variables.  \\
 &  & ptil (K)  & Ice/liquid potential temperature & state1 refers to the instantaneous wind at time $t$  \\
 &  & T (K)   & Temperature & ($t-1$ in Leap Frog), state2 at time $t$ \\
 &  & qv (kg/kg) & Vapor mixing fraction & Diagnostic \\
 &  & qt (kg/kg) & Total water mixing fraction & Prognostic \\ 
 \hline
{\bf shared\_hydro} & nx*ny*nz*nhydro  & hydromtr\%n (1/kg) & number density & Parameters for the hydrometeor populations  \\
 &  & hydromtr\%q (kg/kg) & mass density & (droplet, rain, ice, graupel, snow)  \\
 &  & hydro\_second\%lambda & parameter of the size distribution &  \\
 &  & hydro\_second\%vp & terminal fall speeds &  \\
 &  & hydro\_second\%ifp & presence of hydrometeors (integer) &  \\ 
 &  & hydro\_param\%cm, & Coefficients for microphysics laws &  \\
 &  & ctv, am, btv, nu, cvp1, cvp2 &  &  \\
 \hline 
{\bf shared\_nuc} & nx*ny*nz  & ccn (kg/kg) & mass density of CCNs & Mass densities of CCNs and INs if the  \\
 &  & in (kg/kg) & mass density of INs & aerosol module is not used  \\
 \hline 
{\bf shared\_surf} & nx*ny  & rain (kg/m2) & rain flux at ground level & Horizontal fluxes calculated at ground  \\
 &  & grau (kg/m2) & graupel flux at ground level & level for different precipitating hydrometeors  \\
 &  & snow (kg/m2) & snow flux at ground level &  \\
 &  & ustar (m/s)  & Surface friction velocity \\
 \hline
 {\bf shared\_thermo} & nx*ny*nz  & pp (-) & Total Exner pressure & Diagnostic thermodynamics properties, used  \\
 &  & pt (K)  & potential temperature & mainly by the microphysics scheme  \\
 &  & ptv (K)   & virtual potential temperature & thermo1 is evaluated using state1 variables \\
 &  & sat (kg/kg) & Absolute supersaturation over liquid & \\
 &  & sati (kg/kg) & Absolute supersaturation over ice & \\
 &  & qsw (kg/kg) & saturation vapor content (water) &  \\
 &  & qsi (kg/kg) & saturation vapor content (ice) &  \\ 
 &  & dens (kg/m3)  & Local density & Calculated using perfect gas law \\
 \hline
 {\bf shared\_turbu} & nx*ny*nz  & fk (m2/s2) & Momentum eddy diffusivity &  \\
 &  & fh (m2/s2) & Scalar eddy diffusivity & $fh = fk/Pr$   \\
 &  & kres (m2/s2)  & resolved TKE & related to the dynamics and turbulent model: resolved   \\
 &  & ksgs (m2/s2)   & SGS TKE & and SGS quantities \\
 &  & def (kg/kg) & squared deformation tensor &   \\
 &  & N2 (1/s) & Brunt-Vaisala frequency &  \\
 &  & buoy  & Buoyancy & Vertical acceleration \\
 &  & bfres  & Buoyancy production of TKE (resolved) & Diagnostic \\
 &  & bfsgs  & Buoyancy production of TKE (SGS) & Diagnostic \\
 &  & wvar & Variance of vertical velocity & \\
 &  & wske & Skewness of vertical velocity & \\
 \hline 
 {\bf shared\_rad} & nx*ny*nwz  & dtnet (K/m3) & net temperature tendency & through simplified or detailed model  \\
 &  & fluxsu (K/m2)  & upwind shortwave radiative flux &   \\
 &  & fluxsd (K/m2)   & downwind shortwave radiative flux & \\
 &  & fluxiru (K/m2)  & upwind longwave radiative flux &  \\
 &  & fluxird (K/m2)   & downwind longwave radiative flux &  \\
 &  & frad (K/m2) & net radiative flux & through simplified or detailed model \\
 \hline
\end{tabular}
\end{center}
\caption{{\it Description of the different data structures. For all the above structures, an array {\it xxxx\_mean} is declared containing the mean
structure quantities for output purposes.}}
\label{data_struct}
\end{table}
\end{landscape}


\clearpage
\bibliography{biblio}
\bibliographystyle{plain}

\end{document}
